%%%%%%%%%%%%%%%%%%%%%%%%%%%%%%%%%%%%%%%%%%%%%%%%%%%%%%%%%%%%%%%%%
\begin{slide}{Atmospheric component of HadGEM (New Dynamics)}
%%%%%%%%%%%%%%%%%%%%%%%%%%%%%%%%%%%%%%%%%%%%%%%%%%%%%%%%%%%%%%%%%

\begin{list0}

\item Designed to be more efficient and more accurate than HadCM3

\item Equations solved
\begin{list1}
\item Non-hydrostatic Euler equations
\item Deep Atmosphere
\item 3D Coriolis
\end{list1}

\item Spatial Discretisation -- some changes from HadCM3
\begin{list1}
\item Semi-Lagrangian
\item Cubic or quintic finite difference interpolation onto departure points
\item Lat-lon C-grid
\item Charney-Phillips vertical grid
\item Fixed, height based vertical coordinates (no dependence on pressure)
\item No filtering towards the poles
\item Much less explicit diffusion needed
\end{list1}

\item Temporal Discretisation
\begin{list1}
\item {\color{purple}\bf SISL} Semi-implicit semi-Lagrangian
\item Two time level off centred scheme
\end{list1}

\item Exact Conservation of:
\begin{list1}
\item Mass only by employing Eulerian advection of mass
\end{list1}

\end{list0}

\end{slide}

%%%%%%%%%%%%%%%%%%%%%%%%%%%%%%%%%%%%%%%%%%%%%%%%%%%%%%%%%%%%%%%%%
\begin{slide}{HadGEM and New Dynamics Publications}
%%%%%%%%%%%%%%%%%%%%%%%%%%%%%%%%%%%%%%%%%%%%%%%%%%%%%%%%%%%%%%%%%

\ \\
\begin{list0}
\item \bibentry{DCM+05}
\item \bibentry{JDB+06}
\item \bibentry{MRP+06}
\end{list0}

\end{slide}

%%%%%%%%%%%%%%%%%%%%%%%%%%%%%%%%%%%%%%%%%%%%%%%%%%%%%%%%%%%%%%%%%
\begin{slide}{HadGEM -- Reasons for choices of equations solved}
%%%%%%%%%%%%%%%%%%%%%%%%%%%%%%%%%%%%%%%%%%%%%%%%%%%%%%%%%%%%%%%%%

\item Equations solved
\begin{list1}
\item Non-hydrostatic Euler equations and deep atmosphere
\begin{list2}
\item More accurate at higher resolution
\item Not necessarily much more expensive
\end{list2}

\end{list1}
\end{slide}

%%%%%%%%%%%%%%%%%%%%%%%%%%%%%%%%%%%%%%%%%%%%%%%%%%%%%%%%%%%%%%%%%
\begin{slide}{Semi-Lagrangian}
%%%%%%%%%%%%%%%%%%%%%%%%%%%%%%%%%%%%%%%%%%%%%%%%%%%%%%%%%%%%%%%%%

3 Steps in semi-Lagrangian:
\begin{enumerate}
\item Find departure points: where did the flow depart from at the previous time step to arrive at this grid point. \\
Need predictor step then corrector to estimate velocity at the mid-time between previous and current time steps

\item Interpolate all fields $\psi$ (eg $u$, $v$, $w$, $T$, $p$) from fixed grid at old time onto departure points \\
Needs $\ge$ cubic interpolation to reduce atificial diffusion
\item Integrate forcing along trajectories. Then we have new time level values:\\
$\psi^n_i = \psi(\tilde{X}^{n-1}_i) + \int_C F^{n-\frac{1}{2}}_\text{mid-point} dt$

\hspace{-3ex}Why:
\begin{list2}
\item Much larger, accurate time steps possible $\rightarrow$ efficiency $\times 4$!
\item Less need for polar filtering
\item But conservation becomes very difficult.
\end{list2}

\hspace{-3ex}HadGEM
\begin{list2}
\item Cubic or quintic finite difference interpolation onto departure points
\item In order to guarantee conservation of mass to machine precision, Helmholtz equation solved implicitly using Eulerian discretisaion.
\item Global conservation corrections for tracers.
\item $\therefore$ inconsistent advection of mass and tracers :-(
\end{list2}
\end{enumerate}

\end{slide}

%%%%%%%%%%%%%%%%%%%%%%%%%%%%%%%%%%%%%%%%%%%%%%%%%%%%%%%%%%%%%%%%%
\begin{slide}{HadGEM -- Reasons for Discretisation Choices}
%%%%%%%%%%%%%%%%%%%%%%%%%%%%%%%%%%%%%%%%%%%%%%%%%%%%%%%%%%%%%%%%%

\ \\
\begin{list1}

\item Finite difference -- still good for local area model

\item Lat-lon C-grid :-)

\item Charney-Phillips vertical grid\\
$\rightarrow$ better representation of layer clouds, reducing biases in sub-tropical stratocumulus regions

\item Fixed, height based vertical coordinates (no dependence on pressure)
\begin{list2}
\item Height above the ground near the ground. Smoothly varying to flat vertical levels
\item Can no longer used pressure based coordinate because non-hydrostatic pressure may not be monotonic with height
\item May enhance spurious mixing along isobaric surfaces
\end{list2}

\item Much less explicit diffusion needed
\begin{list2}
\item Due to choice of discretisation without computational modes (C-grid and Charney-Phillips)
\item Semi-Lagrangian advection is inherently diffusive
\end{list2}

\end{list1}
\end{slide}

%%%%%%%%%%%%%%%%%%%%%%%%%%%%%%%%%%%%%%%%%%%%%%%%%%%%%%%%%%%%%%%%%
\begin{slide}{HadGEM Problems}
%%%%%%%%%%%%%%%%%%%%%%%%%%%%%%%%%%%%%%%%%%%%%%%%%%%%%%%%%%%%%%%%%

\ \\
\begin{list0}
\item Tracers not conserved
\item Cannot scale to 10,000 processors. Why not?

\begin{minipage}{0.6\linewidth}
\includegraphics[width=\linewidth]{../../../pLinks/graphics/meshes+latLon+constant+mesh.pdf}

\vspace{-0.55\linewidth} \hspace{2.1in}
\includegraphics[scale=1]{figs/arrow.pdf}

\vspace{0.5\linewidth}\ \\
\end{minipage}
\hfill
\begin{minipage}{0.35\linewidth}

\begin{list1}
\item lat-lon grid?
\item semi-implicit?
\item semi-Lagrangian?
\end{list1}

\end{minipage}


\end{list0}
\end{slide}

