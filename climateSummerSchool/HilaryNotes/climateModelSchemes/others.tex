\begin{slide}{Homework: Find out about another model}

What numerical schemes are used in the atmospheric component of one of these models and why:

\begin{list0}

\item CGCM3.1 from the Canadian Centre for Climate Modelling and Analysis

\item CNRM-CM3 from M\'et\'eo-France/Centre National de Recherches M\'et\'eorologiques

\item CCSM3 from the National Center for Atmospheric Research, USA

\item MIROC3.2 from Japan

\item GFDL-CM2.1 from NOAA/GFDL, USA

This task may not be as easy as it looks. Find out about the basic numerical schemes in the atmosphere model rather than about the parameterisations and resolution. In particular find out:

\begin{list1}
\item What horizontal and vertical discretisation techniques are used? Eg:
\begin{list2}
    \item Eulerian or semi-Lagrangian
    \item Finite difference or spectral
    \item Arakawa grid type
    \item Vertical staggering type
    \item Vertical coordinates
    \item Layout of the horizontal grid -- if lat-lon, how are the poles treated?
\end{list2}

\item What equations are solved (for momentum and continuity)?
\item Which terms are treated implicitly and which explicitly?
\item What limits the time step?
\item What properties are conserved exactly and is this due to inherent conservation properties of the numerical schemes or to conservation corrections?
\item What explicit diffusion or filtering is needed and why?
\end{list1}
\end{list0}

\end{slide}
