
%
%	NCAR summer colloquium on Numerical Methods
%       for Global Atmospheric Models
% 
%       Basic Dynamics - I
%
%	John Thuburn
%	7 April 2008
%
%
%\documentclass[a4,landscape]{slides}
%\documentclass[a4,scdouble]{seminar}

\documentclass[a4]{seminar}
\input seminar.bug

\usepackage{amsmath,amssymb}
\usepackage[dvips]{graphics,color,graphicx}
%\usepackage{psfrag}
%\usepackage{epsfig}
%\usepackage{psfig}
%\usepackage{pstcol}
\usepackage{semlcmss,semcolor}
\usepackage{fancyhdr}

\renewcommand{\labelitemi}{*}

\newcommand{\re}{\mbox{Re}}
\newcommand{\im}{\mbox{Im}}
\newcommand{\ii}{\mathrm{i}}

\newcommand{\R}[1]{\textcolor{red}{#1}}
\newcommand{\B}[1]{\textcolor{blue}{#1}}

%% Bold symbol macro for standard LaTeX users
% \newcommand{\boldsymbol}[1]{\mbox{\boldmath $#1$}}

\definecolor{gray}{rgb}{0.8,0.95,0.95}
\definecolor{lightgray}{rgb}{0.85,0.97,0.97}
%\definecolor{verylightgray}{rgb}{0.89,0.985,0.985}
%\definecolor{verylightgray}{rgb}{0.905,0.985,0.985}
\definecolor{verylightgray}{rgb}{0.985,0.985,0.985}


\pagecolor{verylightgray}

\fancyhead{}
\renewcommand{\headrulewidth}{0.2mm}
\renewcommand{\footrulewidth}{0.2mm}
\fancyhead[L]{\tiny  Basic Dynamics-I
\newline John Thuburn }
%\fancyhead[R]{\mbox{\psfig{file=uelogo.eps,width=15mm}}}
\fancyhead[R]{\makebox{\resizebox{15mm}{!}{\includegraphics{exeterlogo.eps}}}}
%\fancyhead[R]{\tiny University of Exeter}
\fancyfoot[R]{\tiny Page \theslide}
\fancyfoot[C]{}

\setlength{\slidewidth}{250mm}
\setlength{\leftmargin}{30mm}

\renewcommand{\slideleftmargin}{30mm}
\renewcommand{\slidetopmargin}{25mm}
\renewcommand{\headwidth}{\textwidth}


\begin{document}

\slideframe{none}

\pagestyle{fancy}


% --------------------------------------------------------------------------

\begin{slide}

\begin{center}

{\Large \bf
\B{Basic Dynamics - I}
}

\end{center}

\vspace{6mm}

\begin{center}
{\small Monday 2 June, 2008}
\end{center}

\end{slide}

% -------------------------------------------------------------------------

\begin{slide}

\B{\bf Bibliography}

\vspace{6mm}

\(\bullet\) An Introduction to Dynamic Meteorology, J.R. Holton,
Elsevier / Academic Press

\(\bullet\) Atmosphere Ocean Dynamics, A.E. Gill, Academic Press

\(\bullet\) Geophysical Fluid Dynamics, J. Pedlosky, Springer

\(\bullet\) Lectures on Geophysical Fluid Dynamics, R. Salmon, OUP


\end{slide}

% -------------------------------------------------------------------------

\begin{slide}


\B{\bf Outline}

\vspace{4mm}

\(\bullet\) The multiscale nature of atmospheric dynamics

\(\bullet\) Dynamics in a rotating frame

\(\bullet\) Governing equations

\vspace{4mm}

\(\bullet\) Acoustic waves

\(\bullet\) Inertio-gravity waves

\(\bullet\) Phase velocity and group velocity

\vspace{4mm}

\(\bullet\) Some key conservation properties


\end{slide}

% --------------------------------------------------------------------------

\begin{slide}

\B{\bf The multiscale nature of atmospheric dynamics}

\vspace{2mm}

\begin{center}
\makebox{\resizebox{80mm}{!}{{\includegraphics{multiscale2.epsi}}}}
\end{center}


\end{slide}

% --------------------------------------------------------------------------

\begin{slide}

\B{\bf Dynamics in a rotating frame}

\vspace{6mm}

\begin{minipage}[t]{50mm}
\makebox{\resizebox{45mm}{!}{{\includegraphics{rotate.epsi}}}}
\end{minipage}
\ \ \ \ \ \ \ 
\begin{minipage}[b]{50mm}
Let
\vspace{3mm}

\( \mathbf{A} = A_x \mathbf{I} + A_y \mathbf{J} + A_z \mathbf{K} \) \\

\(  \ \ \ \ \ = a_x \mathbf{i} + a_y \mathbf{j} + a_z \mathbf{k} \) 
\end{minipage}

\end{slide}

% --------------------------------------------------------------------------

\begin{slide}

\B{\bf Time rate of change of an arbitrary vector}

\vspace{6mm}


\begin{eqnarray}
\left( \frac{D \mathbf{A}}{Dt} \right)_{\mathrm{IF}}
&
=
&
\mathbf{I} \frac{D A_x}{Dt} +
\mathbf{J} \frac{D A_y}{Dt} +
\mathbf{K} \frac{D A_z}{Dt}
\nonumber \\
&
=
&
\mathbf{i} \frac{D a_x}{Dt} +
\mathbf{j} \frac{D a_y}{Dt} +
\mathbf{k} \frac{D a_z}{Dt}
\nonumber \\
&
+
&
a_x \left( \frac{D \mathbf{i}}{Dt} \right)_{\mathrm{IF}} +
a_y \left( \frac{D \mathbf{j}}{Dt} \right)_{\mathrm{IF}} +
a_z \left( \frac{D \mathbf{k}}{Dt} \right)_{\mathrm{IF}}
\nonumber \\
&
=
&
\left( \frac{D \mathbf{A}}{Dt} \right)_{\mathrm{RF}}
+
\boldsymbol{\Omega} \times \mathbf{A}
\nonumber
\end{eqnarray}

\vspace{3mm}




\end{slide}

% --------------------------------------------------------------------------

\begin{slide}

\B{\bf Apply to position vector}

\vspace{3mm}

\begin{displaymath}
\mathbf{u}_{\mathrm{IF}}
=
\mathbf{u}_{\mathrm{RF}} + \boldsymbol{\Omega} \times \mathbf{x}
\end{displaymath}

\vspace{6mm}

\B{\bf Apply to velocity vector}

\vspace{3mm}

\begin{displaymath}
\mathbf{a}_{\mathrm{IF}}
=
\mathbf{a}_{\mathrm{RF}} + 2 \boldsymbol{\Omega} \times \mathbf{u}_{\mathrm{RF}}
+ \boldsymbol{\Omega} \times ( \boldsymbol{\Omega} \times \mathbf{x} )
\end{displaymath}

\vspace{3mm}


\end{slide}

% --------------------------------------------------------------------------

\begin{slide}

\B{\bf Governing equations}

\vspace{3mm}

Mass

\begin{displaymath}
\frac{\partial \rho}{\partial t}  + \nabla . (\rho \mathbf{u})= 0
\end{displaymath}

\vspace{3mm}

Thermodynamics

\begin{displaymath}
\frac{D \theta}{Dt} = Q
\end{displaymath}

\vspace{3mm}

Momentum

\begin{displaymath}
\frac{D \mathbf{u}}{D t} +
2 \boldsymbol{\Omega} \times \mathbf{u}
=
- \frac{1}{\rho} \nabla p 
- \nabla \Phi
+ \mathbf{F}
\end{displaymath}


\end{slide}

% -------------------------------------------------------------------------

\begin{slide}

along with

\begin{displaymath}
p = R T \rho
\end{displaymath}

\vspace{2mm}

and

\begin{displaymath}
T = \left( \frac{p}{p_{00}} \right)^\kappa \theta
= \Pi (p) \theta
\end{displaymath}

\vspace{2mm}

where \( \kappa = R / C_p \).


\end{slide}

% -------------------------------------------------------------------------

\begin{slide}

\B{\bf Some common approximations}

\vspace{6mm}

\( \bullet \) Spherical geoid: \( \Phi = \Phi(r) \)

\vspace{2mm}

\( \bullet \) Quasi-hydrostatic: neglect \( Dw/Dt\)

\vspace{2mm}

\( \bullet \) Anelastic: \( \nabla ( \rho_0 \mathbf{u}) = 0 \) 

\vspace{2mm}

\( \bullet \) Shallow atmosphere: 

\ \ neglect Coriolis terms involving horizontal component of \( \boldsymbol{\Omega} \);

\ \ replace \( 1/r \) by \( 1/a \);

\ \ neglect some metric terms.



\end{slide}

% -------------------------------------------------------------------------

\begin{slide}

\begin{center}
\makebox{\resizebox{72mm}{!}{{\includegraphics{hierarchy.eps}}}}
\end{center}


\end{slide}

% -------------------------------------------------------------------------

\begin{slide}

\B{\bf The importance of waves}

\vspace{6mm}

\( \bullet \) Acoustic waves: very fast, energetically very weak

\vspace{2mm}

\( \bullet \) Inertio-gravity waves: fast, energetically weak

\vspace{2mm}

\( \bullet \) Rossby waves and balanced vortical motion: energetically dominant

\vspace{5mm}

\R{\bf But} fast waves are crucial for \R{\bf adjustment} towards balance.


\end{slide}

% -------------------------------------------------------------------------

\begin{slide}

\B{\bf Acoustic waves}

\vspace{5mm}

Neglect Coriolis and gravity

\begin{displaymath}
\frac{\partial \rho}{\partial t}  + \nabla . (\rho \mathbf{u})= 0
\end{displaymath}

\begin{displaymath}
\frac{D \mathbf{u}}{D t}
=
- \frac{1}{\rho} \nabla p 
\end{displaymath}


\end{slide}

% -------------------------------------------------------------------------

\begin{slide}

Linearize about an isothermal state of rest

\begin{displaymath}
\frac{\partial \rho}{\partial t}  + \rho_0 \nabla . (\mathbf{u})= 0
\end{displaymath}

\begin{displaymath}
\frac{\partial \mathbf{u}}{\partial t}
=
- \frac{1}{\rho_0} \nabla p
=
- \frac{c^2}{\rho_0} \nabla \rho
\end{displaymath}

where \( c^2 =  \partial p /\partial \rho |_\theta = R T_0 / (1 - \kappa)\)

\vspace{3mm}

\begin{displaymath}
\frac{\partial^2 \rho}{\partial t^2}  - c^2 \nabla^2 \rho = 0
\end{displaymath}

\end{slide}

% -------------------------------------------------------------------------

\begin{slide}

Seek wavelike solutions
\( \propto \exp\{ \ii \mathbf{k} . \mathbf{x} - \omega t\}\)

to obtain the \R{\bf dispersion relation}

\begin{displaymath}
\omega^2 = c^2 | \mathbf{k} |^2
\end{displaymath}

\vspace{3mm}

Note \( \mathbf{u} || \mathbf{k} \): waves are \R{\bf longitudinal}

\vspace{2mm}

Also, acoustic waves are \R{\bf non-dispersive}


\end{slide}

% -------------------------------------------------------------------------

\begin{slide}


\B{\bf Inertio-gravity waves}

\vspace{3mm}

Make the Boussinesq approximation: assume the fluid to be incompresible
\( \nabla . \mathbf{u} = 0 \), and neglect variations in density
except where they appear in a buoyancy term, i.e. multiplied by \( g \).

Also neglect Coriolis terms involving the horizontal component
of \( \boldsymbol{\Omega} \).

\end{slide}

% -------------------------------------------------------------------------

\begin{slide}

\begin{displaymath}
\label{B1}
\frac{Du}{Dt} -fv  = - \frac{1}{\rho_0} \frac{\partial p}{\partial x},
\end{displaymath}
\begin{displaymath}
\label{B2}
\frac{Dv}{Dt} +fu  = - \frac{1}{\rho_0} \frac{\partial p}{\partial y},
\end{displaymath}
\begin{displaymath}
\label{B3}
\frac{Dw}{Dt}  = - \frac{1}{\rho_0} \frac{\partial p}{\partial z} + b,
\end{displaymath}
\begin{displaymath}
\label{B4}
\frac{\partial u}{\partial x} +
\frac{\partial v}{\partial y} +
\frac{\partial w}{\partial z} = 0,
\end{displaymath}
\begin{displaymath}
\label{B6}
\frac{D b}{D t} + w N^2 = 0,
\end{displaymath}

where

\begin{displaymath}
b = - g \frac{\rho'}{\rho_0}
\ \ \ \ \ 
\mathrm{and}
\ \ \ \ \ 
N^2 = - \frac{g}{\rho_0} \frac{\partial \rho_0}{\partial z}
\end{displaymath}

\end{slide}

% -------------------------------------------------------------------------

\begin{slide}

Linearize about a hydrostatic state of rest

\begin{displaymath}
\label{LB1}
\frac{\partial u}{\partial t} - f v = - \frac{1}{\rho_0} \frac{\partial p}{\partial x},
\end{displaymath}
\begin{displaymath}
\label{LB2}
\frac{\partial v}{\partial t} + f u = - \frac{1}{\rho_0} \frac{\partial p}{\partial y},
\end{displaymath}
\begin{displaymath}
\label{LB3}
\frac{\partial w}{\partial t} = - \frac{1}{\rho_0} \frac{\partial p}{\partial z} + b,
\end{displaymath}
\begin{displaymath}
\label{LB4}
\frac{\partial u}{\partial x} +
\frac{\partial v}{\partial y} +
\frac{\partial w}{\partial z} = 0,
\end{displaymath}
\begin{displaymath}
\label{LB6}
\frac{\partial b}{\partial t} + w N^2 = 0.
\end{displaymath}


\end{slide}

% -------------------------------------------------------------------------

\begin{slide}

Seek wavelike solutions \( \propto e^{\ii (kx+ly+mz - \omega t)} \)
to obtain the \R{\bf dispersion relation}
\begin{displaymath}
\omega^2 = \frac{(k^2 + l^2) N^2 + m^2 f^2}{k^2 + l^2 + m^2}
\end{displaymath}

\vspace{3mm}

Note \( \mathbf{u} \perp \mathbf{k} \): waves are \R{\bf transverse}

\vspace{3mm}

For very deep waves, \( m^2 / (k^2 + l^2) \ll 1\), \( \omega^2 \approx N^2 \)

\vspace{3mm}

For very shallow waves, \( (k^2 + l^2) / m^2 \ll 1\), \( \omega^2 \approx f^2 \)




\end{slide}

% -------------------------------------------------------------------------

\begin{slide}

\B{\bf Phase velocity}

If a wavelike disturbance is \( \propto \exp \{\ii \phi(\mathbf{x},t) \} \)

wave crests and troughs are surfaces of constant phase \( \phi \).

For a plane wave

\begin{eqnarray}
\phi & = & \mathbf{k}.\mathbf{x} - \omega t \nonumber \\
     & = & \mathbf{k}.(\mathbf{x} - \mathbf{c}_p t) \nonumber
\end{eqnarray}

where \( \mathbf{c}_p = \mathbf{k} \omega / |\mathbf{k}|^2 \)
is the \R{\bf phase velocity}. Crests and troughs move at velocity
\( \mathbf{c}_p \).

\end{slide}

% -------------------------------------------------------------------------

\begin{slide}

\B{\bf Group velocity}

\vspace{3mm}

How does a \B{\bf packet} of waves propagate?

Consider a superposition of two 1D waves with similar wavenumber and
frequency, satisfying \( \omega = \omega(k) \)

\begin{eqnarray}
q & = & \frac{1}{2} \left(
e^{\ii \{ (k + \delta k) x - (\omega + \delta \omega) t\} }
+
e^{\ii \{ (k - \delta k) x - (\omega - \delta \omega) t\} }
\right) \nonumber \\
 & = &
\cos (\delta k \, x - \delta \omega \, t)
e^{\ii \{ k x - \omega t\} }
\end{eqnarray}


\begin{center}
\makebox{\resizebox{70mm}{!}{{\includegraphics{group_schematic.epsi}}}}
\end{center}


\end{slide}

% -------------------------------------------------------------------------

\begin{slide}

Individual crests and troughs propagate at the \R{\bf phase velocity}

\begin{displaymath}
c_p = \omega / k
\end{displaymath}

but wave packets propagate at the \R{\bf group velocity}

\begin{displaymath}
c_g = \delta \omega / \delta k \rightarrow \partial \omega / \partial k
\end{displaymath}

In 3D

\begin{displaymath}
\mathbf{c}_g = \nabla_{\mathbf{k}} \omega
= \left( \frac{\partial \omega}{\partial k} ,
         \frac{\partial \omega}{\partial l} ,
         \frac{\partial \omega}{\partial m} \right)
\end{displaymath}


\end{slide}

% -------------------------------------------------------------------------

\begin{slide}

\begin{center}
\B{Matlab demo of phase and group velocity}
\end{center}

\end{slide}

% ----------------------------------------------------------------------------

\begin{slide}

\B{\bf Some key conservation properties}

\vspace{3mm}

Some conservation properties can be expressed as

\begin{displaymath}
\frac{\partial \mathcal{A}}{\partial t}  + \nabla . (\mathbf{F})= 0
\end{displaymath}


\vspace{2mm}

\B{\bf Mass}
\begin{displaymath}
\mathcal{A} = \rho \ \ \ \ \ 
\mathrm{and} \ \ \ \ \  \mathbf{F} = \rho \mathbf{u}
\end{displaymath}

\vspace{2mm}

\B{\bf Angular momentum}
\begin{displaymath}
\mathcal{A} = \rho \widehat{\mathbf{z}}.\left[ \mathbf{r}\times
\left( \mathbf{u}+\boldsymbol{\Omega} \times \mathbf{r}\right) \right] \ \ \ \ \ 
\mathrm{and} \ \ \ \ \   
\mathbf{F} = \mathbf{u} \mathcal{A}+p\widehat{\mathbf{z}}\times \mathbf{r}
\end{displaymath}

\vspace{2mm}

\B{\bf Energy}
\begin{displaymath}
\mathcal{A} = \rho (\mathbf{u}^2 / 2 + C_v T + \Phi) \ \ \ \ \ 
\mathrm{and} \ \ \ \ \ 
\mathbf{F} = \mathbf{u}(\mathcal{A} + p)
\end{displaymath}




\end{slide}

% ----------------------------------------------------------------------------

\begin{slide}

\B{\bf Potential temperature}
\begin{displaymath}
\frac{D \theta}{Dt} = 0
\end{displaymath}

\vspace{2mm}

\B{\bf Potential vorticity}
\begin{displaymath}
\frac{DQ}{Dt} = 0
\end{displaymath}
where \( Q = \boldsymbol{\zeta}.\nabla \theta / \rho \)

\vspace{2mm}

\B{\bf Potential enstrophy}
\begin{displaymath}
\mathcal{A} = \rho Q^2 / 2 \ \ \ \ \ 
\mathrm{and} \ \ \ \ \  \mathbf{F} = \mathbf{u} \mathcal{A}
\end{displaymath}


\end{slide}




% ----------------------------------------------------------------------------



% ----------------------------------------------------------------------------



% ----------------------------------------------------------------------------



% ----------------------------------------------------------------------------



% ----------------------------------------------------------------------------



% ----------------------------------------------------------------------------






\end{document}



%%%%%%%%%%%%%%%%%%%%%%%%%%%%%%%%%%%%%%%%%%%%%%%%%%%%%%%%%%%%%%%%%%%%%%%%%
