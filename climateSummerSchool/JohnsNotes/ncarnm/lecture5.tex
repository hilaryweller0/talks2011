

%
%	NCAR summer colloquium on Numerical Methods
%       for Global Atmospheric Models
% 
%       Conservation
%
%	John Thuburn
%	10 May 2008
%
%
%\documentclass[a4,landscape]{slides}
%\documentclass[a4,scdouble]{seminar}

\documentclass[a4]{seminar}
\input seminar.bug

\usepackage{amsmath,amssymb}
\usepackage[dvips]{graphics,color,graphicx}
%\usepackage{psfrag}
%\usepackage{epsfig}
%\usepackage{psfig}
%\usepackage{pstcol}
\usepackage{semlcmss,semcolor}
\usepackage{fancyhdr}

\renewcommand{\labelitemi}{*}

\newcommand{\re}{\mbox{Re}}
\newcommand{\im}{\mbox{Im}}
\newcommand{\ii}{\mathrm{i}}

\newcommand{\R}[1]{\textcolor{red}{#1}}
\newcommand{\B}[1]{\textcolor{blue}{#1}}

%% Bold symbol macro for standard LaTeX users
% \newcommand{\boldsymbol}[1]{\mbox{\boldmath $#1$}}

\definecolor{gray}{rgb}{0.8,0.95,0.95}
\definecolor{lightgray}{rgb}{0.85,0.97,0.97}
%\definecolor{verylightgray}{rgb}{0.89,0.985,0.985}
%\definecolor{verylightgray}{rgb}{0.905,0.985,0.985}
\definecolor{verylightgray}{rgb}{0.985,0.985,0.985}


\pagecolor{verylightgray}

\fancyhead{}
\renewcommand{\headrulewidth}{0.2mm}
\renewcommand{\footrulewidth}{0.2mm}
\fancyhead[L]{\tiny  Conservation
\newline John Thuburn }
%\fancyhead[R]{\mbox{\psfig{file=uelogo.eps,width=15mm}}}
\fancyhead[R]{\makebox{\resizebox{15mm}{!}{\includegraphics{exeterlogo.eps}}}}
%\fancyhead[R]{\tiny University of Exeter}
\fancyfoot[R]{\tiny Page \theslide}
\fancyfoot[C]{}

\setlength{\slidewidth}{250mm}
\setlength{\leftmargin}{30mm}

\renewcommand{\slideleftmargin}{30mm}
\renewcommand{\slidetopmargin}{25mm}
\renewcommand{\headwidth}{\textwidth}


\begin{document}

\slideframe{none}

\pagestyle{fancy}


% --------------------------------------------------------------------------

\begin{slide}

\begin{center}

{\Large \bf
\B{Conservation}
}

\B{\bf in numerical model dynamical cores}

\end{center}

\vspace{6mm}

\begin{center}
{\small Friday 6 June, 2008}
\end{center}

\end{slide}

% -------------------------------------------------------------------------

\begin{slide}


\B{\bf Outline}

\vspace{4mm}

\(\bullet\) Conservation properties of the continuous
adiabatic frictionless governing equations

\(\bullet\) What conservation properties can we obtain in numerical models?

\(\bullet\) Which conservation properties are most relevant/important?

\ \ \ \ Finite resolution effects; the adiabatic frictionless \R{\bf limit}

\ \ \ \ Respecting appropriate asymptotic limits

\ \ \ \ Spurious sources vs physical sources

\vspace{4mm}

(Closely following T 2008, J. Comput. Phys.)

\end{slide}

% --------------------------------------------------------------------------

\begin{slide}

\B{\bf Conservation properties of the continuous
adiabatic frictionless governing equations}

\vspace{4mm}

Flux form conservation laws

\vspace{3mm}

Lagrangian conservation laws

\vspace{3mm}

Conserved integral quantities

\vspace{3mm}

Kinematic identities


\end{slide}

% --------------------------------------------------------------------------

\begin{slide}

\B{\bf Flux form conservation laws}

\begin{displaymath}
\frac{\partial A}{\partial t}+\nabla .\mathbf{F}=0,
\end{displaymath}

\vspace{3mm}

\begin{tabular}{|l|cc|}
\hline
Quantity
&
\( A \)
&
\( \mathbf{F} \)
\\
\hline
\hline
\B{\bf Mass}
&
\( \rho \)
&
\( \rho \mathbf{u} \)
\\
\hline
\B{\bf Angular momentum}
&
\( \rho \widehat{\mathbf{z}}.\left[ \mathbf{r}\times
\left( \mathbf{u}+\boldsymbol{\Omega} \times \mathbf{r}\right) \right] \)
&
\( \mathbf{u}A+p\widehat{\mathbf{z}}\times \mathbf{r} \)
\\
\hline
\B{\bf Energy}
&
\( \rho \left( \frac{1}{2}\mathbf{u}^{2}+c_{v}T+\Phi \right)  \)
&
\( \mathbf{u}\left( A+p\right) \)
\\
\hline
\end{tabular}



\end{slide}

% --------------------------------------------------------------------------

\begin{slide}

\B{\bf Lagrangian conservation laws}

\begin{displaymath}
\frac{D \chi}{D t} = 0 \ \ \Rightarrow \ \ \frac{D f(\chi)}{D t} = 0
\end{displaymath}

\vspace{2mm}

\begin{tabular}{|l|}
\hline
\B{\bf Potential temperature} \( \theta \) \\
\B{\bf Potential vorticity} \( Q = \boldsymbol{\zeta} . \nabla \theta / \rho \) \\
\B{\bf Specific tracer} \( q \) or \B{\bf tracer mixing ratio} \( \eta \) \\
\hline
\end{tabular}

\vspace{2mm}

Each Lagrangian conservation law generates an infinite family of
flux form conservation laws
\begin{displaymath}
\frac{\partial \rho f(\chi)}{\partial t}+
\nabla .\left( \rho \mathbf{u} f(\chi) \right) = 0
\end{displaymath}


\end{slide}

% --------------------------------------------------------------------------

\begin{slide}

\B{\bf Conserved integral quantities}

\begin{tabular}{|p{45mm}c|}
\hline
\B{\bf Mass per unit \( \theta \)} in an isentropic layer
&
\( \mathcal{F}(\theta )=\int \rho /\left| \nabla \theta \right| dA \)
\\
\B{\bf Mass per unit \( \theta \)} in an isentropic layer within a material contour
&
\( \mathcal{M}=\int_{D}\rho /\left| \nabla \theta \right| dA \)
\\
\B{\bf Absolute circulation} around an isentropic material contour
&
\( \mathcal{C}=\oint_{\Gamma }\mathbf{v}_{a}.d\mathbf{r}
=\int_{D}\rho Q/\left| \nabla \theta \right| dA \)
\\
\hline
\end{tabular}


\end{slide}

% --------------------------------------------------------------------------

\begin{slide}

\B{\bf Kinematic identities}

\vspace{2mm}

The global integrals of horizontal divergence
\begin{displaymath}
\int_{D} \delta \, dA
\end{displaymath}
and vertical component of vorticity
\begin{displaymath}
\int_{D} \zeta \, dA
\end{displaymath}
must vanish
on any isosurface of the vertical coordinate that wraps the sphere.


\end{slide}

% --------------------------------------------------------------------------

\begin{slide}

\B{\bf Techniques for obtaining or approximating conservation
properties in numerical models}

\vspace{3mm}

\B{\bf 1 Predict the desired variable using a discrete 
flux form conservation law}
\begin{displaymath}
\frac{A_j^{n+1} - A_j^{n}}{\Delta t} + 
\frac{F_{j+1/2} - F_{j-1/2}}{\Delta x_j} = 0
\end{displaymath}


E.g.\ \( A \) might be \( \rho \) times specific humidity.



\end{slide}

% -------------------------------------------------------------------------

\begin{slide}

\B{\bf Techniques for obtaining or approximating conservation
properties in numerical models}

\vspace{3mm}

\B{\bf 2 Impose discrete analogues of special cancellations}

\vspace{3mm}

E.g.\ Coriolis terms on the C-grid; Arakawa Jacobian

\vspace{3mm}

In some cases there are systematic ways of deriving such schemes
using Poisson bracket and Nambu bracket ideas

\vspace{3mm}

(May only work globally)


\end{slide}

% -------------------------------------------------------------------------

\begin{slide}

\B{\bf Techniques for obtaining or approximating conservation
properties in numerical models}

\vspace{3mm}

\B{\bf 3 Lagrangian conservation properties}

\vspace{2mm}

Use a Lagrangian solution technique

\vspace{2mm}

Use Lagrangian or quasi-Lagrangian coordinates
(or at least vertical coordinate)

\vspace{2mm}

Nonoscillatory advection schemes (perhaps combined with `reverse engineering')


\end{slide}

% --------------------------------------------------------------------------

\begin{slide}

\B{\bf Techniques for obtaining or approximating conservation
properties in numerical models}

\vspace{3mm}

\B{\bf 4 Special forms of scale-selective dissipation}

\vspace{3mm}

E.g.\ Anticipated potential vorticity method

\vspace{3mm}

E.g.\ Energy backscatter

\end{slide}

% -------------------------------------------------------------------------

\begin{slide}

\B{\bf Which conservation properties are most relevant/important?}

\vspace{2mm}

\B{\bf Finite resolution effects}

Take \( \rho \equiv 1 \), let \( \chi \) be specific tracer.
Define
\begin{displaymath}
V_i = \int_{\mathrm{cell}\ i} \, dV
\end{displaymath}
\begin{displaymath}
m_i = \left( \int_{\mathrm{cell}\ i} \chi \, dV \right) / V_i
\end{displaymath}
\begin{displaymath}
r_i = \left( \int_{\mathrm{cell}\ i} \chi^2 \, dV \right) / V_i
\end{displaymath}
Then
\begin{displaymath}
\int \chi dV = \Sigma_i m_i V_i
\end{displaymath}
But
\begin{displaymath}
\int \chi^2 dV = \Sigma_i r_i V_i \geq \Sigma_i m_i^2 V_i
\end{displaymath}



\end{slide}

% -------------------------------------------------------------------------

\begin{slide}

\B{\bf Which conservation properties are most relevant/important?}

\vspace{2mm}

\B{\bf Adiabatic and frictionless, or adiabatic frictionelss {\it limit} ?}

\vspace{3mm}

Quantities like tracer variance and potential enstrophy that cascade
downscale are dissipated even in the limit of vanishing viscosity and
thermal diffusivity. They are \R{\bf non-Robust} invariants

\vspace{2mm}

\R{\bf Dilemma}: attempt to conserve non-robust invariants, then dissipate
them with a \B{\bf sub-grid model}, or use inherently dissipative
numerical methods \B{\bf (ILES)}.

\vspace{2mm}

\begin{center}
\B{Matlab demo of enstrophy conservation}
\end{center}

\vspace{3mm}

What about \R{\bf energy}?


\end{slide}

% -------------------------------------------------------------------------
\begin{slide}

\B{\bf Energy}

\vspace{2mm}

Total energy (\( \sim 3 \times 10^9 \mathrm{Jm}^{-2} \)) is made up of
available and unavailable contributions

\begin{center}
\makebox{\resizebox{70mm}{!}{{\includegraphics{available2.epsi}}}}

\vspace{2mm}

\begin{tabular}{|c|ccc|}
\hline

&
Unavailable PE
&
Available PE
&
KE
\\
Ratio:
&
\( 2000 \)
&
\( 4 \)
&
\( 1 \)
\\
\hline
\end{tabular}

\end{center}

\end{slide}

% -------------------------------------------------------------------------

\begin{slide}

Unavailable and available energy are separately conserved

\vspace{3mm}

Unavailable energy is a function of the \( \mathcal{F}(\theta) \) - almost robust

\vspace{3mm}

Can conserve mass in each model isentropic layer by using \( \theta \) as
a vertical coordinate

\vspace{3mm}

There is some evidence that about 5-10\% of available energy
cascades downscale in the free atmosphere (the rest goes upscale before
being dissipated by the boundary layer)



\end{slide}

% -------------------------------------------------------------------------

\begin{slide}

\B{\bf Which conservation properties are most relevant/important?}

\vspace{2mm}

\B{\bf Respecting asymptotic limits (Mike Cullen)} 

\vspace{2mm}

There is no mathematical proof that solutions of the continuous
governing equations exist, but there are for certain relevant
asymptotic limits (e.g.\ semi-geostrophic equations).

\vspace{2mm}

It is argued that the properties that are essential for the existence
proof (e.g.\ boundedness of PV under advection) are also the properties
that \B{\bf control} the dynamics, and should therefore be respected
by numerical methods.

\vspace{2mm}
This also argues for numerical methods respecting the mixed hyperbolic-elliptic
nature of the asymptotic limits.




\end{slide}

% -------------------------------------------------------------------------

\begin{slide}


\B{\bf Which conservation properties are most relevant/important?}

\vspace{2mm}

\B{\bf Spurious sources vs physical sources} 

\vspace{3mm}

Our numerical solutions should be accurate provided
spurious numerical sources of conservable quantities
are much weaker than true physical sources.

\vspace{2mm}

Conveniently expressed in terms of timescales.



\end{slide}

% -------------------------------------------------------------------------

\begin{slide}

\B{\bf Mass}

\vspace{3mm}

Essentially no sources of dry air:

\begin{displaymath}
\tau \sim \infty
\end{displaymath}


\end{slide}

% -------------------------------------------------------------------------

\begin{slide}

\B{\bf Momentum and angular momentum}

\vspace{3mm}

Locally, adjustment towards balance is fast

\( \tau \sim \) few 10s of seconds to
few 10s of hours

\vspace{3mm}

\B{\bf But}, in a zonal mean, terms in the \( u \) equation
are not in geostrophic balance.

Global mean angular momentum
\( \sim \pm 0.4 \times 10^{26} \, \mathrm{kg}\,\mathrm{m}^2 \mathrm{s}^{-1} \)

Typical surface torque
\( \sim \pm 0.5 \times 10^{20} \, \mathrm{kg}\,\mathrm{m}^2 \mathrm{s}^{-2} \)

\( \tau \sim 10 \, \mathrm{days} \)

\vspace{2mm}

(But locally much longer, e.g.\ QBO.)


\end{slide}

% -------------------------------------------------------------------------

\begin{slide}

\B{\bf Tracer variance}

\vspace{2mm}

Estimates of ``mixdown time'' suggest \( \tau \sim 10-20 \, \mathrm{days} \)


\vspace{3mm}

\B{\bf Potential enstrophy}

\vspace{2mm}

Enstrophy budgets suggest \( \tau \sim 10 \, \mathrm{days} \)



\end{slide}

% -------------------------------------------------------------------------

\begin{slide}

\B{\bf Unavailable energy}

\vspace{2mm}

Global mean \( \sim 3 \times 10^9 \, \mathrm{Jm}^{-2} \)

Total energy throughput of climate system \( \sim 240 \, \mathrm{Wm}^{-2} \)

\( \tau \sim 150 \, \mathrm{days} \)



\end{slide}

% -------------------------------------------------------------------------

\begin{slide}

\B{\bf Available energy}

Global mean \( \sim 6 \times 10^6 \, \mathrm{Jm}^{-2} \)

Available energy throughput of the atmosphere \( \sim 2 \, \mathrm{Wm}^{-2} \)

\( \tau \sim 30 \, \mathrm{days} \)



\end{slide}

% -------------------------------------------------------------------------

\begin{slide}


\B{\bf Summary}

\vspace{2mm}

\begin{tabular}{|l|l|l|l|}
\hline
Quantity    & Robust     & Cascade & Approx.\ timescale\\
\hline
\hline
Mass        & Yes        &         & Infinite \\  
\hline
Momentum    &            &         & Minutes to hours \\
Angular momentum &       &         & 10~days (locally longer) \\
\hline
Potential enstrophy &    & Yes     & 10~days \\
Tracer variance     &    & Yes     & 10~days \\
\hline
Unavailable energy & Almost &      & 150~days \\
Available energy   &     & Yes (5-10\%) & 20-30~days \\
\hline
Entropy      & Almost   &         & Variable \\
\hline
\end{tabular}




\end{slide}

% -------------------------------------------------------------------------

\begin{slide}

\B{\bf Concluding remark}

\vspace{2mm}

There is no perfect general purpose method for solving the compressible
Navier-Stokes equations.

The methods developed by weather and climate modellers
aim to capture the most important aspects of the physics
\B{\bf for flow regime of interest}

and to exploit flow properties for efficiency.

\end{slide}

% -------------------------------------------------------------------------




% ----------------------------------------------------------------------------






\end{document}



%%%%%%%%%%%%%%%%%%%%%%%%%%%%%%%%%%%%%%%%%%%%%%%%%%%%%%%%%%%%%%%%%%%%%%%%%
