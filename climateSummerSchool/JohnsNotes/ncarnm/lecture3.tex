
%
%	NCAR summer colloquium on Numerical Methods
%       for Global Atmospheric Models
% 
%       Horizontal discretizations
%
%	John Thuburn
%	8 May 2008
%
%
%\documentclass[a4,landscape]{slides}
%\documentclass[a4,scdouble]{seminar}

\documentclass[a4]{seminar}
\input seminar.bug

\usepackage{amsmath,amssymb}
\usepackage[dvips]{graphics,color,graphicx}
%\usepackage{psfrag}
%\usepackage{epsfig}
%\usepackage{psfig}
%\usepackage{pstcol}
\usepackage{semlcmss,semcolor}
\usepackage{fancyhdr}

\renewcommand{\labelitemi}{*}

\newcommand{\re}{\mbox{Re}}
\newcommand{\im}{\mbox{Im}}
\newcommand{\ii}{\mathrm{i}}

\newcommand{\R}[1]{\textcolor{red}{#1}}
\newcommand{\B}[1]{\textcolor{blue}{#1}}

%% Bold symbol macro for standard LaTeX users
% \newcommand{\boldsymbol}[1]{\mbox{\boldmath $#1$}}

\definecolor{gray}{rgb}{0.8,0.95,0.95}
\definecolor{lightgray}{rgb}{0.85,0.97,0.97}
%\definecolor{verylightgray}{rgb}{0.89,0.985,0.985}
%\definecolor{verylightgray}{rgb}{0.905,0.985,0.985}
\definecolor{verylightgray}{rgb}{0.985,0.985,0.985}


\pagecolor{verylightgray}

\fancyhead{}
\renewcommand{\headrulewidth}{0.2mm}
\renewcommand{\footrulewidth}{0.2mm}
\fancyhead[L]{\tiny  Horizontal discretizations
\newline John Thuburn }
%\fancyhead[R]{\mbox{\psfig{file=uelogo.eps,width=15mm}}}
\fancyhead[R]{\makebox{\resizebox{15mm}{!}{\includegraphics{exeterlogo.eps}}}}
%\fancyhead[R]{\tiny University of Exeter}
\fancyfoot[R]{\tiny Page \theslide}
\fancyfoot[C]{}

\setlength{\slidewidth}{250mm}
\setlength{\leftmargin}{30mm}

\renewcommand{\slideleftmargin}{30mm}
\renewcommand{\slidetopmargin}{25mm}
\renewcommand{\headwidth}{\textwidth}


\begin{document}

\slideframe{none}

\pagestyle{fancy}


% --------------------------------------------------------------------------

\begin{slide}

\begin{center}

{\Large \bf
\B{Horizontal discretizations}
}

\end{center}

\vspace{6mm}

\begin{center}
{\small Wednesday 4 June, 2008}
\end{center}

\end{slide}

% -------------------------------------------------------------------------

\begin{slide}

\B{\bf Outline}

\vspace{4mm}

\(\bullet\) Wave propagation and staggered grids

\ \ \ \ 1D gravity waves

\ \ \ \ 2D inertio-gravity waves

\ \ \ \ 2D Rossby waves

\vspace{4mm}

\(\bullet\) Conservation properties


\vspace{6mm}
(I will not discuss the spectral method, nor advection schemes)


\end{slide}

% --------------------------------------------------------------------------

\begin{slide}

\B{\bf Wave propagation and staggered grids}

\vspace{3mm}

Motivation: accurate representation of propagation of \R{\bf fast waves} and
hence \R{\bf adjustment} and \R{\bf balance}.

\end{slide}

% --------------------------------------------------------------------------

\begin{slide}

\B{\bf 1D gravity waves}

\vspace{6mm}

Linearized, 1D, non-rotating, shallow water equations

\begin{displaymath}
\frac{\partial \Phi}{\partial t} + \Phi_0 \frac{\partial u}{\partial x} = 0
\ \ \ \ \ 
\frac{\partial u}{\partial t} + \frac{\partial \Phi}{\partial x} = 0
\end{displaymath}

\vspace{2mm}

Look for wavelike solutions
\begin{displaymath}
\Phi = Re \left\{ \hat{\Phi} \exp[\ii(kx - \omega t)]\right\}
\ \ \ \ \ 
u = Re \left\{ \hat{u} \exp[\ii(kx - \omega t)]\right\}
\end{displaymath}

to find the dispersion relation
\begin{displaymath}
\omega^2 = k^2 \Phi_0
\end{displaymath}


\end{slide}

% --------------------------------------------------------------------------

\begin{slide}

\B{\bf On an unstaggered grid}

\vspace{2mm}

\begin{displaymath}
\frac{\partial u_j}{\partial t} + \frac{\Phi_{j+1} - \Phi_{j-1}}{2 \Delta x} = 0
\ \ \ \ \ \ 
\frac{\partial \Phi_j}{\partial t} + \frac{u_{j+1} - u_{j-1}}{2 \Delta x} = 0
\end{displaymath}

\vspace{3mm}
\hspace{20mm} \( k \rightarrow \sin (k \Delta x) / \Delta x \)
\vspace{3mm}

\begin{center}
\makebox{\resizebox{85mm}{!}{{\includegraphics{unstag2.epsi}}}}
\end{center}



\end{slide}

% --------------------------------------------------------------------------

\begin{slide}

\B{\bf On a staggered grid}

\vspace{2mm}

\begin{displaymath}
\frac{\partial u_{j+1/2}}{\partial t} + \frac{\Phi_{j+1} - \Phi_{j}}{\Delta x} = 0
\ \ \ \ \ \ 
\frac{\partial \Phi_j}{\partial t} + \frac{u_{j+1/2} - u_{j-1/2}}{\Delta x} = 0
\end{displaymath}

\vspace{3mm}
\hspace{20mm} \( k \rightarrow \sin (k \Delta x / 2) / ( \Delta x / 2) \)
\vspace{3mm}

\begin{center}
\makebox{\resizebox{85mm}{!}{{\includegraphics{stag2.epsi}}}}
\end{center}



\end{slide}

% -------------------------------------------------------------------------

\begin{slide}

\begin{center}
\B{\bf Matlab demo of unstaggered and staggered grids}
\end{center}

\end{slide}

% -------------------------------------------------------------------------

\begin{slide}

\B{\bf Computational modes}

Spuriously fail to propagate

\vspace{4mm}

\B{\bf Parasitic modes}

Spurious propagation, e.g.\ wrong sign of group velocity

Implications for inhomogeneous and adaptive grids

\begin{center}
\B{\bf Matlab demo of parasitic mode}
\end{center}


\end{slide}

% -------------------------------------------------------------------------

\begin{slide}

\B{\bf 2D inertio-gravity waves}

Linearized, 2D, rotating, shallow water equations

\begin{eqnarray}
\frac{\partial \Phi}{\partial t} +
\Phi_0 \left( \frac{\partial u}{\partial x} + \frac{\partial v}{\partial y} \right)
& = & 0 \nonumber \\
\frac{\partial u}{\partial t} - fv + \frac{\partial \Phi}{\partial x}
& = & 0 \nonumber \\
\frac{\partial v}{\partial t} + fu + \frac{\partial \Phi}{\partial y}
& = & 0 \nonumber
\end{eqnarray}

\vspace{3mm}

Dispersion relation (for \( f = f_0 = \mathrm{const}\))
\begin{displaymath}
\omega \left( \omega^2 - f_0^2 - (k^2 + l^2)\Phi_0 \right) = 0
\end{displaymath}


\end{slide}

% -------------------------------------------------------------------------

\begin{slide}

\B{\bf The Rossby radius}

Define

\begin{displaymath}
\lambda = \Phi_0^{1/2} / f_0
\end{displaymath}

\vspace{2mm}
On scales smaller than \( \lambda \) pressure gradient terms dominate.

\vspace{2mm}
On scales greater than \( \lambda \) Coriolis terms dominate.


\end{slide}

% -------------------------------------------------------------------------

\begin{slide}

\B{\bf A-grid}

\begin{minipage}{47mm}
\makebox{\resizebox{45mm}{!}{{\includegraphics{gridA.eps}}}}
\end{minipage}
\begin{minipage}{60mm}
\( k \rightarrow \sin (k \Delta x) / \Delta x \)

\( l \rightarrow \sin (l \Delta y) / \Delta y \)

\( f_0 \rightarrow f_0 \)
\end{minipage}

\begin{center}
\makebox{\resizebox{80mm}{!}{{\includegraphics{dispgridA.eps}}}}
\end{center}

\end{slide}

% -------------------------------------------------------------------------

\begin{slide}

\B{\bf B-grid}

\begin{minipage}{47mm}
\makebox{\resizebox{45mm}{!}{{\includegraphics{gridB.eps}}}}
\end{minipage}
\begin{minipage}{60mm}
\( k \rightarrow \cos(l \Delta y / 2) \sin (k \Delta x / 2) / ( \Delta x / 2 )\)

\( l \rightarrow \cos(k \Delta x / 2) \sin (l \Delta y / 2) / ( \Delta y / 2 )\)

\( f_0 \rightarrow f_0 \)
\end{minipage}


\begin{center}
\makebox{\resizebox{80mm}{!}{{\includegraphics{dispgridB.eps}}}}
\end{center}

\end{slide}

% -------------------------------------------------------------------------

\begin{slide}

\B{\bf C-grid}

\begin{minipage}{47mm}
\makebox{\resizebox{45mm}{!}{{\includegraphics{gridC.eps}}}}
\end{minipage}
\begin{minipage}{60mm}
\( k \rightarrow \sin (k \Delta x / 2) / ( \Delta x / 2 )\)

\( l \rightarrow \sin (l \Delta y / 2) / ( \Delta y / 2 )\)

\( f_0 \rightarrow f_0 \cos(k \Delta x / 2) \cos(l \Delta y / 2)\)
\end{minipage}


\begin{center}
\makebox{\resizebox{80mm}{!}{{\includegraphics{dispgridC.eps}}}}
\end{center}

\end{slide}

% -------------------------------------------------------------------------

\begin{slide}

\begin{minipage}{44mm}
\B{\bf D-grid}

\makebox{\resizebox{42mm}{!}{{\includegraphics{gridD.eps}}}}
\end{minipage}
\begin{minipage}{44mm}
\B{\bf E-grid}

\makebox{\resizebox{42mm}{!}{{\includegraphics{gridE.eps}}}}
\end{minipage}

\vspace{2mm}

\begin{minipage}{44mm}
\B{\bf Z-grid}

\makebox{\resizebox{42mm}{!}{{\includegraphics{gridZ.eps}}}}
\end{minipage}


\end{slide}

% -------------------------------------------------------------------------

\begin{slide}


\begin{center}
\B{\bf Matlab demo of geostrophic adjustment}
\end{center}


\end{slide}

% -------------------------------------------------------------------------

\begin{slide}

There exist analogues of these staggered grids for other grid cell shapes
such as triangles and hexagons - see the lecture by Todd Ringler.

\vspace{4mm}

Also, analogous ideas hold for finite-element discretizations.

\end{slide}

% -------------------------------------------------------------------------

\begin{slide}

\B{\bf Rossby wave propagation on the C-grid}

\vspace{2mm}

Does averaging of Coriolis terms lead to poor representation of
Rossby wave propagation?

\begin{minipage}{47mm}
\makebox{\resizebox{45mm}{!}{{\includegraphics{gridC.eps}}}}
\end{minipage}
\begin{minipage}{50mm}
When \( f \) is a function of position there are various ways of
averaging the Coriolis terms
\end{minipage}


\end{slide}

% -------------------------------------------------------------------------

\begin{slide}

\R{\( f \)-at-\(\Phi\)-points} is quite good:

\begin{displaymath}
\partial_t u - \overline{f \overline{v}^y}^x + \delta_x \Phi = 0
\end{displaymath}
\begin{displaymath}
\partial_t v + \overline{f \overline{u}^x}^y + \delta_y \Phi = 0
\end{displaymath}

It captures the wave frequency quite well even for short north-south wavelengths
(but not short east-west wavelengths).

\vspace{2mm}

It is also energy conserving.


\end{slide}

% -------------------------------------------------------------------------

\begin{slide}

\begin{center}
\makebox{\resizebox{80mm}{!}{{\includegraphics{f_at_u2.epsi}}}}
\end{center}

\end{slide}

% -------------------------------------------------------------------------


\begin{slide}

\B{\bf In spherical geometry} it is important to include appropriate
geometrical factors in the averaging of the Coriolis terms, for consistency
with the mass continuity equation:

\begin{displaymath}
\frac{\partial u}{\partial t}
-
\overline{\frac{f}{\cos \phi} \overline{v \cos \phi}^\phi}^\lambda
+
\frac{1}{a \cos \phi} \delta_\lambda \Phi
=0
\end{displaymath}

\begin{displaymath}
\frac{\partial v}{\partial t}
+
\overline{f \overline{u}^\lambda}^\phi
+
\frac{1}{a } \delta_\phi \Phi
=0
\end{displaymath}




\end{slide}

% -------------------------------------------------------------------------

\begin{slide}
\begin{center}
\makebox{\resizebox{50mm}{!}{{\includegraphics{f_at_u_sphere2.epsi}}}}
\makebox{\resizebox{50mm}{!}{{\includegraphics{f_at_u_sphere_spurious2.epsi}}}}
\end{center}

\end{slide}

% -------------------------------------------------------------------------

\begin{slide}

\B{\bf Energy conservation - Coriolis terms}

\vspace{2mm}

Coriolis terms should cancel when we take \( u \) times
\begin{displaymath}
\frac{D u}{D t} - f v = \ldots
\end{displaymath}
plus \( v \) times
\begin{displaymath}
\frac{D v}{D t} + f u = \ldots
\end{displaymath}

\vspace{2mm}

Straightforward for A-grid and B-grid






\end{slide}

% -------------------------------------------------------------------------

\begin{slide}

For the C-grid, write \( u^{\mathrm{*}} = u \Phi a \Delta \phi \),
\( v^{\mathrm{*}} = v \Phi a \cos \phi \Delta \lambda \)

\begin{displaymath}
\begin{array}{l}
\frac{\partial }{\partial t}( u a \cos \phi \Delta \lambda )_{i,j+1/2} \\
- \alpha_{i,j+1/2} \, v^{\mathrm{*}}_{i+1/2,j+1}
-  \beta_{i,j+1/2} \, v^{\mathrm{*}}_{i-1/2,j+1} \\
- \gamma_{i,j+1/2} \, v^{\mathrm{*}}_{i-1/2,j}
- \delta_{i,j+1/2} \, v^{\mathrm{*}}_{i+1/2,j} = \ldots
\end{array}
\end{displaymath}

\begin{displaymath}
\begin{array}{l}
\frac{\partial }{\partial t}( v a \Delta \phi )_{i+1/2,j} \\
+ \alpha_{i,j-1/2} \, u^{\mathrm{*}}_{i,j-1/2}
+ \beta_{i+1,j-1/2} \, u^{\mathrm{*}}_{i+1,j-1/2} \\
+ \gamma_{i+1,j+1/2} \, u^{\mathrm{*}}_{i+1,j+1/2}
+ \delta_{i,j+1/2} \, u^{\mathrm{*}}_{i,j+1/2} = \ldots
\end{array}
\end{displaymath}



\end{slide}

% -------------------------------------------------------------------------

\begin{slide}

\B{\bf Energy conservation - pressure gradient terms}

\vspace{2mm}

We require a discrete analogue of
\begin{displaymath}
\mathbf{v} . \nabla p + p \nabla . \mathbf{v} = \nabla . (\mathbf{v} p)
\end{displaymath}
or, at least,
\begin{displaymath}
\int \mathbf{v} . \nabla p \, dA + \int p \nabla . \mathbf{v} \, dA = 0
\end{displaymath}

\vspace{2mm}

Relatively straightforward on the A-grid.


\end{slide}

% -------------------------------------------------------------------------

\begin{slide}

E.g.\ on a C-grid write
\( \hat{u} = u a \Delta \phi = u^{\mathrm{*}} / \rho \),
\( \hat{v} = v  a \cos \phi \Delta \lambda = v^{\mathrm{*}} / \rho \)

Then
\begin{displaymath}
\begin{array}{l}
\Sigma_{i \, j} \, \hat{u}_{i,j+1/2} \,
\left( p_{i+1/2,j+1/2} - p_{i-1/2,j+1/2} \right) + \\
\Sigma_{i \, j} \, \hat{v}_{i+1/2,j} \,
\left( p_{i+1/2,j+1/2} - p_{i+1/2,j-1/2} \right) + \\
\Sigma_{i \, j} \, p_{i+1/2,j+1/2} \,
\left( \hat{u}_{i+1,j+1/2} - \hat{u}_{i,j+1/2} \right) + \\
\Sigma_{i \, j} \, p_{i+1/2,j+1/2} \,
\left( \hat{v}_{i+1/2,j+1} - \hat{v}_{i+1/2,j} \right) = 0 \\
\end{array}
\end{displaymath}

(with care at the poles)

\end{slide}

% -------------------------------------------------------------------------

\begin{slide}

Similar considerations apply to \B{\bf angular momentum conservation}.

\vspace{6mm}

\B{\bf conservation of potential enstrophy} is also possible, at least
in the shallow water case, but is more complicated.



\end{slide}

% -------------------------------------------------------------------------

\begin{slide}


\end{slide}

% ----------------------------------------------------------------------------

\begin{slide}



\end{slide}

% ----------------------------------------------------------------------------

\begin{slide}


\end{slide}




% ----------------------------------------------------------------------------



% ----------------------------------------------------------------------------



% ----------------------------------------------------------------------------



% ----------------------------------------------------------------------------



% ----------------------------------------------------------------------------



% ----------------------------------------------------------------------------






\end{document}



%%%%%%%%%%%%%%%%%%%%%%%%%%%%%%%%%%%%%%%%%%%%%%%%%%%%%%%%%%%%%%%%%%%%%%%%%
