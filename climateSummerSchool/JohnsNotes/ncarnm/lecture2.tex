
%
%	NCAR summer colloquium on Numerical Methods
%       for Global Atmospheric Models
% 
%       Basic Dynamics - II
%
%	John Thuburn
%	13 April 2008
%
%
%\documentclass[a4,landscape]{slides}
%\documentclass[a4,scdouble]{seminar}

\documentclass[a4]{seminar}
\input seminar.bug

\usepackage{amsmath,amssymb}
\usepackage[dvips]{graphics,color,graphicx}
%\usepackage{psfrag}
%\usepackage{epsfig}
%\usepackage{psfig}
%\usepackage{pstcol}
\usepackage{semlcmss,semcolor}
\usepackage{fancyhdr}

\renewcommand{\labelitemi}{*}

\newcommand{\re}{\mbox{Re}}
\newcommand{\im}{\mbox{Im}}
\newcommand{\ii}{\mathrm{i}}

\newcommand{\R}[1]{\textcolor{red}{#1}}
\newcommand{\B}[1]{\textcolor{blue}{#1}}

%% Bold symbol macro for standard LaTeX users
% \newcommand{\boldsymbol}[1]{\mbox{\boldmath $#1$}}

\definecolor{gray}{rgb}{0.8,0.95,0.95}
\definecolor{lightgray}{rgb}{0.85,0.97,0.97}
%\definecolor{verylightgray}{rgb}{0.89,0.985,0.985}
%\definecolor{verylightgray}{rgb}{0.905,0.985,0.985}
\definecolor{verylightgray}{rgb}{0.985,0.985,0.985}


\pagecolor{verylightgray}

\fancyhead{}
\renewcommand{\headrulewidth}{0.2mm}
\renewcommand{\footrulewidth}{0.2mm}
\fancyhead[L]{\tiny  Basic Dynamics-II
\newline John Thuburn }
%\fancyhead[R]{\mbox{\psfig{file=uelogo.eps,width=15mm}}}
\fancyhead[R]{\makebox{\resizebox{15mm}{!}{\includegraphics{exeterlogo.eps}}}}
%\fancyhead[R]{\tiny University of Exeter}
\fancyfoot[R]{\tiny Page \theslide}
\fancyfoot[C]{}

\setlength{\slidewidth}{250mm}
\setlength{\leftmargin}{30mm}

\renewcommand{\slideleftmargin}{30mm}
\renewcommand{\slidetopmargin}{25mm}
\renewcommand{\headwidth}{\textwidth}


\begin{document}

\slideframe{none}

\pagestyle{fancy}


% --------------------------------------------------------------------------

\begin{slide}

\begin{center}

{\Large \bf
\B{Basic Dynamics - II}
}

\end{center}

\vspace{6mm}

\begin{center}
{\small Tuesday 3 June, 2008}
\end{center}

\end{slide}

% -------------------------------------------------------------------------

\begin{slide}


\B{\bf Outline}

\vspace{4mm}

\(\bullet\) Hydrostatic balance

\(\bullet\) Geostrophic balance

\(\bullet\) Quasigeostrophic theory; Rossby waves

\vspace{4mm}

\(\bullet\) Eulerian and Lagrangian timescales

\vspace{4mm}

\(\bullet\) Turbulence; energy and potential enstrophy cascades


\end{slide}

% --------------------------------------------------------------------------

\begin{slide}

\B{\bf Hydrostatic balance}

\vspace{3mm}

Scale analysis of vertical momentum equation (following Holton)

\vspace{1mm}

\begin{center}
\begin{tabular}{|lccccc|}
\hline
\(w\)-equation
&
\(\frac{Dw}{Dt}\)
&
\(-\frac{u^2 + v^2}{r}\)
&
\(-2 \Omega u \cos \phi\)
&
\(g\)
&
\(\frac{1}{\rho} \frac{\partial p}{\partial r}\)
\\
Scales
&
\(UW/L\)
&
\(U^2 / a\)
&
\(f_0 U\)
&
\(g\)
&
\(P_0 / \rho H \)
\\
Values \( \mathrm{ms}^{-2}\)
&
\(10^{-7}\)
&
\(10^{-5}\)
&
\(10^{-3}\)
&
\(10\)
&
\(10\)
\\
\hline
\end{tabular}
\end{center}

\vspace{3mm}

Clearly
\begin{displaymath}
g + \frac{1}{\rho} \frac{\partial p}{\partial r} \approx 0
\end{displaymath}


\end{slide}

% --------------------------------------------------------------------------

\begin{slide}

\B{\bf Geostrophic balance}

\vspace{3mm}

Scale analysis of horizontal momentum equation (following Holton)

\vspace{1mm}

\begin{center}
\begin{tabular}{|lcccccc|}
\hline
\(u\)-equation
&
\(\frac{Du}{Dt}\)
&
\(-\frac{u v \tan \phi}{r}\)
&
\(\frac{u w}{r}\)
&
\(-2 \Omega v \sin \phi\)
&
\(2 \Omega w \cos \phi\)
&
\(\frac{1}{\rho r \cos \phi} \frac{\partial p}{\partial \lambda}\)
\\
\(v\)-equation
&
\(\frac{Dv}{Dt}\)
&
\(-\frac{u^2 \tan \phi}{r}\)
&
\(\frac{v w}{r}\)
&
\(2 \Omega u \sin \phi\)
&
&
\(\frac{1}{\rho r} \frac{\partial p}{\partial \phi}\)
\\
Scales
&
\(U^2/L\)
&
\(U^2 / a\)
&
\( UW/a\)
&
\(f_0 U\)
&
\(f_0 W\)
&
\(\delta P / \rho L \)
\\
Values \( \mathrm{ms}^{-2}\)
&
\(10^{-4}\)
&
\(10^{-5}\)
&
\(10^{-8}\)
&
\(10^{-3}\)
&
\(10^{-6}\)
&
\(10^{-3}\)
\\
\hline
\end{tabular}
\end{center}

\vspace{3mm}

Then
\begin{displaymath}
u \approx - \frac{1}{f_0 \rho r} \frac{\partial p}{\partial \phi} \equiv u_g;
\ \ \ \ \ \ 
v \approx \frac{1}{f_0 \rho r \cos \phi} \frac{\partial p}{\partial \lambda} \equiv v_g 
\end{displaymath}

\vspace{3mm}

\begin{displaymath}
\mathrm{Rossby\ number:} \ \ \ \ Ro = U/(f_0 L)
\end{displaymath} 


\end{slide}

% --------------------------------------------------------------------------

\begin{slide}

\B{\bf Conditions for validity of hydrostatic approximation}

\vspace{2mm}

Can neglect \R{\(Dw / Dt\)} when
\begin{displaymath}
\frac{UW}{L} \ll \frac{\delta P}{\rho H}
\end{displaymath}

From horizontal momentum equation
\begin{displaymath}
\frac{\delta P}{\rho} \sim U^2 \ \ \ \mathrm{or}\ \ \ f_0 L U
\end{displaymath}

So we require
\begin{displaymath}
\frac{WH}{UL} \ll 1 \ \ \ \mathrm{or}\ \ \ \frac{WH}{UL}Ro \ll 1
\end{displaymath}

\end{slide}

% --------------------------------------------------------------------------

\begin{slide}

From the mass continuity equation
\begin{displaymath}
\frac{W}{U} \sim \frac{H}{L} \ \ \ \mathrm{or} \ \ \ \frac{W}{U} \sim \frac{H}{L}Ro
\end{displaymath}

so hydrostatic balance will be a good approximation provided
\begin{displaymath}
\frac{H^2}{L^2} \ll 1 \ \ \ \mathrm{or} \ \ \ \frac{H^2}{L^2}Ro^2 \ll 1
\end{displaymath}

\vspace{2mm}
\R{\bf Alternatively}
\begin{displaymath}
\omega^2 \ll N^2
\end{displaymath}


\end{slide}

% --------------------------------------------------------------------------

\begin{slide}

Hydrostatic \B{\bf balance} corresponds to the \B{\bf absence of certain
kinds of waves}, namely internal acoustic and fast inertio-gravity waves.

\vspace{2mm}

Mathematically, it leads to a \B{\bf nonlocal} 1D boundary value problem for \( w \).

\vspace{2mm}

It can be thought of as an asymptotic limit in which some information
propagates infinitely fast, so part of the
\B{\bf adjustment} process is \B{\bf instantaneous}.

\vspace{2mm}

Numerical methods should be able to capture such important asymptotic
limits.


\end{slide}

% --------------------------------------------------------------------------

\begin{slide}

\B{\bf Sketch of quasigeostrophic theory}

\vspace{3mm}

Work in \(\beta\)-plane geometry \( f = f_0 + \beta y\) and a log-pressure
vertical coordinate \( \tilde{z} \). Assume hydrostatic balance, that \( Ro \ll 1 \),
that thermodynamic quantities are close to reference profiles, and that
\( \beta L / f_0 \ll 1\).

\vspace{2mm}

At leading order \( u \approx u_g \), \( v \approx v_g \).

\vspace{2mm}
\begin{displaymath}
u_g = -\frac{\partial \psi}{\partial y}; \ \ \ \ 
v_g = \frac{\partial \psi}{\partial x}; \ \ \ \ 
\frac{\theta'}{\theta_{\mathrm{ref}}} = \frac{f}{g} \frac{\partial
\psi}{\partial \tilde{z}}
\end{displaymath}
where \( \psi = \Phi' / f_0 \).

\vspace{2mm}

So let
\( u = u_g + u_a \), \( v = v_g + v_a \)


\end{slide}

% -------------------------------------------------------------------------

\begin{slide}

At next order we get a vorticity equation
\begin{displaymath}
\frac{D_g \zeta_g}{Dt} =
\frac{f_0}{\rho_0} \frac{\partial}{\partial \tilde{z}} \left( \rho_0 \tilde{w} \right)
\end{displaymath}

which is combined with the thermodynamic equation
\begin{displaymath}
\frac{D_g \theta'}{Dt} + \tilde{w} \frac{\partial \theta_0}{\partial \tilde{z}} = 0
\end{displaymath}

to form the potential vorticity equation
\begin{displaymath}
\frac{D_g q}{Dt} = 0
\end{displaymath}

where
\begin{displaymath}
q = f_0 + \beta y + \nabla_{\tilde{z}}^2 \psi +
\frac{1}{\rho_0} \frac{\partial}{\partial \tilde{z}}
\left( \rho_0 \frac{f_0^2}{N^2_{\mathrm{ref}}} \frac{\partial \psi}{\partial \tilde{z}} \right) 
\end{displaymath}




\end{slide}

% -------------------------------------------------------------------------

\begin{slide}

\B{\bf Quasigeostrophic theory filters out all fast waves}

\vspace{2mm}

Mathematically, it leads to a \B{\bf nonlocal} 3D boundary value problem for \( \psi \).

\vspace{2mm}

It can be thought of as an asymptotic limit in which some information
propagates infinitely fast, so the
\B{\bf adjustment} process is \B{\bf instantaneous}.

\vspace{2mm}

Again, numerical methods should be able to capture such important asymptotic
limits.


\end{slide}

% --------------------------------------------------------------------------



\begin{slide}

Quasigeostrophic theory embodies \B{\bf advection},
i.e.\ material conservation, and \B{\bf invertibility}
of potential vorticity.

\vspace{2mm}

These two properties can be used to understand many
GFD phenomena

\end{slide}

% -------------------------------------------------------------------------

\begin{slide}

\B{\bf E.g.\ Rossby waves}

Linearize PV advection and invertibility equations about a state of rest:
\begin{displaymath}
\frac{\partial q}{\partial t} + \beta v_g = 0
\end{displaymath}
\begin{displaymath}
q = \nabla_{\tilde{z}}^2 \psi +
\frac{1}{\rho_0} \frac{\partial}{\partial \tilde{z}}
\left( \rho_0 \frac{f_0^2}{N^2_{\mathrm{ref}}} \frac{\partial \psi}{\partial \tilde{z}} \right) 
\end{displaymath}

Seek wavelike solutions
\begin{displaymath}
\psi = Re \left\{ \hat{\psi}(\tilde{z}) \exp \ii (kx + ly + m \tilde{z} - \omega t)\right\}
\end{displaymath}
to obtain
\begin{displaymath}
\omega = - \frac{\beta k}{k^2 + l^2 + (m^2 + 1/4H_\rho^2) f_0^2 / N^2_\mathrm{ref}}
\end{displaymath}


\end{slide}

% -------------------------------------------------------------------------

\begin{slide}

\begin{center}
\B{Matlab demo of Rossby wave}
\end{center}

\end{slide}

% -------------------------------------------------------------------------

\begin{slide}

\B{\bf Eulerian and Lagrangian timescales}

\vspace{3mm}

\begin{center}
\makebox{\resizebox{70mm}{!}{{\includegraphics{lagrange.eps}}}}
\end{center}

\end{slide}

% -------------------------------------------------------------------------

\begin{slide}

Large scale atmospheric flow has a steep energy spectrum, something like
\R{\( k^{-3}\)}

\B{\bf Eulerian timescale}

\begin{displaymath}
\tau_{\mathrm{Eul}} \sim \frac{L}{U} \sim \frac{1}{k U_0}
\end{displaymath}

\B{\bf Lagrangian timescale}

Some quantities are approximately materially conserved
\R{\( D \chi / Dt \approx 0 \)}
and so have long Lagrangian timescale.

For other quantities
\begin{displaymath}
\tau_{\mathrm{Lag}} \sim \frac{1}{S} \sim \frac{1}{k_0 U_0}
\end{displaymath}

\vspace{2mm}

This fact can be exploited by approximating time derivatives in a
Lagrangian way.

\end{slide}

% -------------------------------------------------------------------------

\begin{slide}

\B{\bf However...}

an important exception is flow over orography, for which
\( \tau_{\mathrm{Eul}} \) becomes very long, but
\begin{displaymath}
\tau_{\mathrm{Lag}} \sim \frac{L}{U} \sim \frac{1}{k U_0}
\end{displaymath}

In this case (semi-)Lagrangian methods with long time steps
suffer from spurious \R{\bf orographic resonance}.


\end{slide}

% -------------------------------------------------------------------------

\begin{slide}

\B{\bf Turbulence and cascades}

\vspace{5mm}

\R{\bf Nonlinearity} implies \R{\bf interaction of scales}.

\vspace{2mm}

Dynamics attempts to generate variability at and below the
grid scale.





\end{slide}

% -------------------------------------------------------------------------

\begin{slide}

\B{\bf Kolmogorov (1941) theory}


\begin{center}
\makebox{\resizebox{70mm}{!}{{\includegraphics{cascade.epsi}}}}
\end{center}

For 3D, statistically steady, homogeneous, isotropic turbulence,
in an \B{\bf inertial range}:

At wavenumber \( k \), the only dimensional quantities are
the energy throughput \( \varepsilon \) and \( k \) itself.


\end{slide}

% -------------------------------------------------------------------------

\begin{slide}

\begin{displaymath}
\left[ E(k) \right] = L^3 T^{-2}
\end{displaymath}

\begin{displaymath}
\left[ \varepsilon \right] = L^2 T^{-3} \ \ \ \mathrm{and} \ \ \ 
\left[ k \right] = L^{-1}
\end{displaymath}

so

\begin{displaymath}
E(k) = C_1 \varepsilon^{2/3} k ^{-5/3}
\end{displaymath}

for some universal \( C_1 \) of order \( 1 \).


\end{slide}

% -------------------------------------------------------------------------

\begin{slide}

\B{\bf Two dimensional turbulence}

\vspace{2mm}

In 2D turbulence we have another conservable quantity, the enstrophy,
and therefore a cascade of enstrophy \( \eta \).

Typically energy now cascades upscale while enstrophy cascades
downscale.

\begin{center}
\makebox{\resizebox{70mm}{!}{{\includegraphics{cascade2.epsi}}}}
\end{center}

\end{slide}

% -------------------------------------------------------------------------

\begin{slide}

In the energy cascade region \( E(k) = C \varepsilon^{2/3} k ^{-5/3} \),
as before.

\vspace{2mm}

In the enstrophy cascade region

\begin{displaymath}
\left[ E(k) \right] = L^3 T^{-2}
\end{displaymath}

\begin{displaymath}
\left[ \eta \right] = T^{-3} \ \ \ \mathrm{and} \ \ \ 
\left[ k \right] = L^{-1}
\end{displaymath}

so

\begin{displaymath}
E(k) = C_2 \eta^{2/3} k^{-3}
\end{displaymath}

for some universal \( C_2 \) of order \( 1 \).


\end{slide}

% -------------------------------------------------------------------------

\begin{slide}

\B{\bf Energy upscale, enstrophy downscale (mostly)}

Let
\begin{displaymath}
E = \int E(k) \, dk \ \ \ \mathrm{and} \ \ \ Z = \int k^2 E(k) \, dk
\end{displaymath}

Suppose energy is initially concentrated near wavenumber \( k_1 \) and
subsequently spreads out, so that
\begin{displaymath}
\frac{d}{dt} \int (k - k_1)^2 E(k) \, dk > 0
\end{displaymath}

The fact that \( E \) and \( Z \) are conserved (neglecting viscosity)
implies
\begin{displaymath}
\frac{d}{dt} \left( \frac{\int k E(k) \, dk}{\int E(k) \, dk} \right) < 0
\end{displaymath}


\end{slide}

% -------------------------------------------------------------------------

\begin{slide}

Similarly, assuming
\begin{displaymath}
\frac{d}{dt} \int (k^2 - k_1^2)^2 E(k) \, dk > 0
\end{displaymath}

implies
\begin{displaymath}
\frac{d}{dt} \left( \frac{\int k^2 Z(k) \, dk}{\int Z(k) \, dk} \right) > 0
\end{displaymath}



\end{slide}

% -------------------------------------------------------------------------

\begin{slide}

\begin{center}
\B{Matlab demo of cascades}
\end{center}

\end{slide}


% ----------------------------------------------------------------------------






\end{document}



%%%%%%%%%%%%%%%%%%%%%%%%%%%%%%%%%%%%%%%%%%%%%%%%%%%%%%%%%%%%%%%%%%%%%%%%%
