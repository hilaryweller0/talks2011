
%
%	NCAR summer colloquium on Numerical Methods
%       for Global Atmospheric Models
% 
%       Vertical discretizations
%
%	John Thuburn
%	18 May 2008
%
%
%\documentclass[a4,landscape]{slides}
%\documentclass[a4,scdouble]{seminar}

\documentclass[a4]{seminar}
\input seminar.bug

\usepackage{amsmath,amssymb}
\usepackage[dvips]{graphics,color,graphicx}
%\usepackage{psfrag}
%\usepackage{epsfig}
%\usepackage{psfig}
%\usepackage{pstcol}
\usepackage{semlcmss,semcolor}
\usepackage{fancyhdr}

\renewcommand{\labelitemi}{*}

\newcommand{\re}{\mbox{Re}}
\newcommand{\im}{\mbox{Im}}
\newcommand{\ii}{\mathrm{i}}

\newcommand{\R}[1]{\textcolor{red}{#1}}
\newcommand{\B}[1]{\textcolor{blue}{#1}}

%% Bold symbol macro for standard LaTeX users
% \newcommand{\boldsymbol}[1]{\mbox{\boldmath $#1$}}

\definecolor{gray}{rgb}{0.8,0.95,0.95}
\definecolor{lightgray}{rgb}{0.85,0.97,0.97}
%\definecolor{verylightgray}{rgb}{0.89,0.985,0.985}
%\definecolor{verylightgray}{rgb}{0.905,0.985,0.985}
\definecolor{verylightgray}{rgb}{0.985,0.985,0.985}


\pagecolor{verylightgray}

\fancyhead{}
\renewcommand{\headrulewidth}{0.2mm}
\renewcommand{\footrulewidth}{0.2mm}
\fancyhead[L]{\tiny  Vertical discretizations
\newline John Thuburn }
%\fancyhead[R]{\mbox{\psfig{file=uelogo.eps,width=15mm}}}
\fancyhead[R]{\makebox{\resizebox{15mm}{!}{\includegraphics{exeterlogo.eps}}}}
%\fancyhead[R]{\tiny University of Exeter}
\fancyfoot[R]{\tiny Page \theslide}
\fancyfoot[C]{}

\setlength{\slidewidth}{250mm}
\setlength{\leftmargin}{30mm}

\renewcommand{\slideleftmargin}{30mm}
\renewcommand{\slidetopmargin}{25mm}
\renewcommand{\headwidth}{\textwidth}


\begin{document}

\slideframe{none}

\pagestyle{fancy}


% --------------------------------------------------------------------------

\begin{slide}

\begin{center}

{\Large \bf
\B{Vertical discretizations}
}

\end{center}

\vspace{6mm}

\begin{center}
{\small Thursday 5 June, 2008}
\end{center}

\end{slide}

% -------------------------------------------------------------------------

\begin{slide}

\B{\bf Outline}

\vspace{4mm}

\(\bullet\) Different vertical coordinates

\vspace{4mm}

\(\bullet\) Bottom and top boundary conditions

\vspace{4mm}

\(\bullet\) Example: the energy and angular momentum conserving scheme
of Simmons and Burridge

\vspace{4mm}

\(\bullet\) Wave dispersion and balance



\end{slide}

% --------------------------------------------------------------------------

\begin{slide}

\B{\bf Different vertical coordinates}

\vspace{3mm}

\begin{minipage}{52mm}
\makebox{\resizebox{50mm}{!}{{\includegraphics{coord.eps}}}}
\end{minipage}
\begin{minipage}{60mm}

Transformation rules:

\begin{displaymath}
\frac{\partial \psi}{\partial z} =
\frac{\partial \eta}{\partial z}\frac{\partial \psi}{\partial \eta}
\end{displaymath}
\begin{displaymath}
\left(\frac{\partial \psi}{\partial s}\right)_z =
\left(\frac{\partial \psi}{\partial s}\right)_\eta +
\left(\frac{\partial \psi}{\partial \eta} \right)
\left(\frac{\partial \eta}{\partial s}\right)_z
\end{displaymath}
\begin{displaymath}
\frac{D \psi}{D t} =
\frac{\partial \psi}{\partial t} +
\mathbf{v} . \nabla_{\mathrm{H}} +
\dot{\eta} \frac{\partial \psi}{\partial \eta}
\end{displaymath}

\end{minipage}

\vspace{3mm}

Note it is usual to retain the usual velocity components
\begin{center}
\( u = \dot{\lambda} a \cos \phi \), \ \ \( v = \dot{\phi} a \),
\ \ \( w = \dot{z}\) ( as well as \( \dot{\eta} \) )
\end{center}



\end{slide}

% --------------------------------------------------------------------------

\begin{slide}

\B{\bf Examples}

\vspace{6mm}

(i) \R{\bf Height} \( \eta = z \)

\vspace{3mm}

(ii) \R{\bf Pressure} \( \eta = p \)

\vspace{3mm}

(iii) \R{\bf Mass} \( \eta = \int_z^\infty g \rho dz \)



\end{slide}

% --------------------------------------------------------------------------

\begin{slide}

\B{\bf Examples}

\vspace{2mm}

\begin{minipage}{50mm}
(iv) \R{\bf Terrain following} variants

E.g.\ \( \eta = z - z_s\),

or \( \eta = p/p_s \)
\end{minipage}
\ \ \ \ \ \ \ \ \ 
\begin{minipage}{50mm}
Also (v) \R{\bf hybrid terrain following} variants

E.g.\ \( \eta = a + b \)

where \( p = a p_0 + b p_s \) 
\end{minipage}

\begin{center}
\makebox{\resizebox{120mm}{!}{{\includegraphics{tfc.eps}}}}
\end{center}



\end{slide}

% --------------------------------------------------------------------------

\begin{slide}

\B{\bf Examples}

\vspace{6mm}

(vi) \R{\bf Isentropic coordinate} \( \eta = f(\theta) \)

\vspace{3mm}

(vii) \R{\bf Lagrangian vertical coordinate} \( \dot{\eta} = 0 \)

\vspace{3mm}


\end{slide}

% --------------------------------------------------------------------------

\begin{slide}

\B{\bf Bottom boundary condition}

\vspace{2mm}

\begin{minipage}{55mm}
\R{\bf No normal flow}
\end{minipage}
\begin{minipage}{55mm}
\begin{center}
\makebox{\resizebox{53mm}{!}{{\includegraphics{bbc.eps}}}}
\end{center}
\end{minipage}

\vspace{2mm}

\( \dot{\eta} = 0 \) if \( \eta \) is terrain following

\vspace{2mm}

\( w = 0 \) if the ground is flat. Otherwise \( w = \mathbf{v} . \nabla_{H} z_s \)

\vspace{2mm}

When a boundary layer is included (\R{\bf no slip}) then
\( \mathbf{v} = \mathbf{0} \), \( w = 0 \)

But for a frictionless dynamical core (\R{\bf free slip})
we need a value of \( \mathbf{v}_s \).



\end{slide}

% -------------------------------------------------------------------------

\begin{slide}

An alternative is to use a \R{\bf terrain intersecting coordinate}

\begin{center}
\makebox{\resizebox{75mm}{!}{{\includegraphics{etabc.eps}}}}
\end{center}

possibly using \R{\bf fractional cells} or \R{\bf shaved cells}

\vspace{2mm}

Tricky when the coordinate surfaces and their intersections with
the ground may move.



\end{slide}

% -------------------------------------------------------------------------

\begin{slide}

\B{\bf Top boundary condition}

\vspace{2mm}

The real atmosphere has no top boundary!

\vspace{3mm}

Practical choices include

\vspace{2mm}

\( \bullet \) \R{\bf rigid lid} at some constant \( z \). \( w = 0 \) there.
Angular momentum and energy conservation are preserved.

\vspace{3mm}

\( \bullet \) \R{\bf elastic lid} at some constant \( p \) (may be \( p = 0 \) ).
\( \dot{p} = 0 \) there. Continuous equations conserve
angular momentum and enthalpy.


\vspace{5mm}

It is common to include some (non-scale-selective) damping near the model
top to reduce \R{\bf spurious wave reflection}.




\end{slide}

% -------------------------------------------------------------------------

\begin{slide}

\B{\bf Example: the Simmons and Burridge
energy and angular momentum conserving scheme}

for the hydrostatic primitive equations

\vspace{3mm}

Hybrid ``sigma-pressure'' coordinate \( p = a p_0 + b p_s \)

\begin{center}
\makebox{\resizebox{55mm}{!}{{\includegraphics{simmons.eps}}}}
\end{center}


\end{slide}

% -------------------------------------------------------------------------

\begin{slide}

\B{\bf Hydrostatic equation}

\begin{displaymath}
\frac{\partial \Phi}{\partial \eta} =
- \frac{R T}{p} \frac{\partial p}{\partial \eta}
\end{displaymath}

naturally becomes

\begin{displaymath}
\Phi_{k+1/2} - \Phi_{k-1/2}
=
- R T_k \ln \frac{p_{k+1/2}}{p_{k-1/2}}
\end{displaymath}

and then

\begin{displaymath}
\Phi_k = \Phi_{k+1/2} +
\alpha_k R T_k
\end{displaymath}

\vspace{2mm}

Note the existence of a \R{\bf `computational mode'}: a non-zero pattern
in \( T \) that is invisible to the \( \Phi_k \).


\end{slide}

% -------------------------------------------------------------------------

\begin{slide}

\B{\bf Angular momentum conservation}

\begin{displaymath}
\frac{D \mathbf{v}}{Dt} + f \hat{\mathbf{z}} \times \mathbf{v}
+ \nabla \Phi +
\frac{R T}{p} \nabla p = 0
\end{displaymath}

Take \( \int_0^1 (\dots) \, \partial p / \partial \eta \, d \eta \)

For angular momentum conservation we require
\begin{displaymath}
\sum_{r=1}^N \Phi_r \frac{\partial}{\partial \lambda} \Delta p_r
=
\Phi_s \frac{\partial p_s}{\partial \lambda}
+
\sum_{r=1}^N R
\left( \frac{T}{p} \frac{\partial p}{\partial \lambda} \right)_r
\Delta p_r
\end{displaymath}

which is satisfied if

\begin{displaymath}
\left( \frac{R T}{p} \nabla p \right)_k
=
\frac{r T_k}{\Delta p_k}
\left[ \left( \ln \frac{p_{k+1/2}}{p_{k-1/2}} \right) \nabla p_{k-1/2}
+ \alpha_k \nabla(\Delta p_k)
\right]
\end{displaymath}



\end{slide}

% -------------------------------------------------------------------------

\begin{slide}


\B{\bf Energy conservation}

\vspace{2mm}

\( \mathbf{v} \) times momentum equation:

\begin{displaymath}
\frac{D}{Dt} \left( \frac{\mathbf{v}^2}{2} \right)
+
\mathbf{v} . \nabla \Phi
+
\frac{R T}{p} \mathbf{v} . \nabla{p}
= 0
\end{displaymath}

Thermodynamic equation
\begin{displaymath}
\frac{D}{Dt} c_p T
-
\frac{R T \omega}{p}
= 0
\end{displaymath}

where
\begin{displaymath}
\omega \equiv \dot{p} =
-\int_0^\eta \nabla .
\left( \mathbf{v} \frac{\partial p}{\partial \eta} \right)
\, d \eta
+
\mathbf{v} . \nabla p
\end{displaymath}



\end{slide}

% -------------------------------------------------------------------------

\begin{slide}

We require

\begin{displaymath}
\int \left( \mathbf{v}.\nabla \Phi + \frac{R T}{p} \mathbf{v}.\nabla p
- \frac{RT \omega}{p} \right) \frac{\partial p}{\partial \eta}
\, d \eta \, dA = 0
\end{displaymath}

This will be satisfied provided we evaluate
\begin{displaymath}
\left( \frac{1}{p} \nabla p \right)_k
\end{displaymath}

as in the momentum equation, and \( RT/p\) times the
vertical integral term as

\begin{displaymath}
\frac{R T_k}{\Delta p_k}
\left[
\left( \ln \frac{p_{k+1/2}}{p_{k-1/2}} \right)
\sum_{r=1}^{k-1} \nabla . (\mathbf{v}_r \Delta p_r )
+
\alpha_k \nabla.(\mathbf{v}_k \Delta p_k)
\right]
\end{displaymath}


\end{slide}

% -------------------------------------------------------------------------

\begin{slide}

\B{\bf Wave propagation, and balance}

\vspace{4mm}

There are issues analogous to those that arise when choosing
horizontal discretizations:

Which choices of prognostic variables
and grid staggering best capture \R{\bf wave propagation},
\R{\bf adjustment}, and \R{\bf balance}?



\end{slide}

% -------------------------------------------------------------------------

\begin{slide}

\B{\bf The Lorenz and Charney-Phillips grids}

for the hydrostatic case

\begin{center}
\makebox{\resizebox{95mm}{!}{{\includegraphics{lcp.eps}}}}
\end{center}


\end{slide}

% -------------------------------------------------------------------------

\begin{slide}

\B{\bf The Lorenz grid computational mode}

\vspace{2mm}

Consider the Simmons and Burridge scheme

Linearize about a state of hydrostatic balance, and
compare neighboring \( \Phi \) values

\begin{displaymath}
\Phi'_{k} - \Phi'_{k-1}
=
( \alpha_k - ( \Delta \ln p)_k) T'_k - \alpha_{k-1} T'_{k-1}
\end{displaymath}

\vspace{2mm}

The left hand side can vanish provided a certain average of the
\( T' \)s vanishes

\vspace{2mm}

There is some oscillatory pattern in \( T_k \) that is invisible
to the \( \Phi_k \), and therefore does not propagate.

This also implies a spurious resonant response to steady forcing.


\end{slide}

% -------------------------------------------------------------------------

\begin{slide}

\B{\bf Compressible Euler equations}

\vspace{3mm}

We now need \R{\bf five} prognostic variables, so there are many
choices of staggering.

Also, there is some freedom over which
thermodynamic variables to use (any two from
\( \rho \), \( p \), \( T \), \( \theta \), etc...)

\vspace{4mm}

Here consider \R{\bf \( z \) coordinate}. Similar reasoning applies to
other vertical coordinates (e.g.\ mass, isentropic).



\end{slide}

% -------------------------------------------------------------------------

\begin{slide}

We want to minimize the use of vertical averaging and of
finite differences over \( 2 \Delta z \)

\vspace{3mm}

Numerical exploration of a large number of cases shows

\vspace{4mm}

\( \bullet \) accurate representation of acoustic waves is necessary
(but not sufficient) for an accurate representation of inertio-gravity
waves

\vspace{4mm}

\( \bullet \) which in turn is necessary (but not sufficient) for
an accurate representation of Rossby waves.


\end{slide}

% -------------------------------------------------------------------------

\begin{slide}

\B{\bf Acoustic waves}

\begin{displaymath}
\omega^2 \approx m^2 c^2
\end{displaymath}


\vspace{3mm}

To capture acoustic waves (reasonably) accurately in the limit
of short vertical wavelength we require

\vspace{2mm}

\( \bullet \) \( \delta_z p \) at the same level as \( w \)

\vspace{2mm}

\( \bullet \) \( \delta_z w \) at the same level as \( p \)

\vspace{4mm}

What if we don't predict \( p \) ?


\end{slide}

% -------------------------------------------------------------------------

\begin{slide}

\B{\bf Inertio-gravity waves}

\begin{displaymath}
\omega^2 \approx \frac{ m^2 f^2 + K^2 N^2 }{m^2 + K^2}
\end{displaymath}


\vspace{3mm}

To capture inertio-gravity waves (reasonably) accurately in the limit
of short vertical wavelength we require

\vspace{2mm}

\( \bullet \) \( u \) and \( v \) at the same level as \( p \)

\vspace{2mm}

\( \bullet \) buoyancy variable at the same level as \( w \)


\vspace{4mm}

What if we don't predict \( p \) ?



\end{slide}


% -------------------------------------------------------------------------

\begin{slide}

\B{\bf Rossby waves}

\begin{displaymath}
\omega \approx - \frac{k \beta N^2}{m^2 f^2 + K^2 N^2}
\end{displaymath}


\vspace{3mm}

To capture Rossby waves (reasonably) accurately in the limit
of short vertical wavelength we require

\vspace{2mm}

\( \bullet \) buoyancy variable at the same level as \( w \)


\end{slide}

% -------------------------------------------------------------------------

\begin{slide}

\B{\bf Example: optimal configuration}

\begin{center}
\makebox{\resizebox{95mm}{!}{{\includegraphics{optimal.eps}}}}
\end{center}

with pressure gradient term written as \( c_p \theta \nabla \Pi \) 


\end{slide}

% -------------------------------------------------------------------------

\begin{slide}

\begin{center}
\makebox{\resizebox{95mm}{!}{{\includegraphics{optimal_disp.ps}}}}
\end{center}


\end{slide}

% -------------------------------------------------------------------------

\begin{slide}

\B{\bf Example: Lorenz-like grid}

\begin{center}
\makebox{\resizebox{95mm}{!}{{\includegraphics{lorenz.eps}}}}
\end{center}

Note computational mode.

\end{slide}

% -------------------------------------------------------------------------

\begin{slide}

\begin{center}
\makebox{\resizebox{95mm}{!}{{\includegraphics{lorenz_disp.ps}}}}
\end{center}

\end{slide}

% -------------------------------------------------------------------------

\begin{slide}

\B{\bf Example: sub-optimal configuration}


\begin{center}
\makebox{\resizebox{95mm}{!}{{\includegraphics{optimal2.eps}}}}
\end{center}

with pressure gradient term written as \( (1 / \rho ) \nabla p \)
and \( p \) diagnosed from \( \rho \) and \( \theta \)


\end{slide}

% -------------------------------------------------------------------------

\begin{slide}

\begin{center}
\makebox{\resizebox{95mm}{!}{{\includegraphics{suboptimal2_disp.ps}}}}
\end{center}


\end{slide}

% -------------------------------------------------------------------------

\begin{slide}

\B{\bf Example: optimal configuration predicting \( \rho \)}


\begin{center}
\makebox{\resizebox{95mm}{!}{{\includegraphics{optimal2.eps}}}}
\end{center}

with pressure gradient term written as \( c_p \theta \nabla \Pi \)
and \( \Pi \) diagnosed from \( \rho \) and \( \theta \)


\end{slide}

% -------------------------------------------------------------------------

\begin{slide}

\begin{center}
\makebox{\resizebox{95mm}{!}{{\includegraphics{optimal2_disp.ps}}}}
\end{center}

\end{slide}

% -------------------------------------------------------------------------

\begin{slide}

Note the \R{\bf apparent incompatibility} between
\B{\bf optimal wave propagation without computational modes}

\vspace{3mm}

and

\vspace{3mm}

\B{\bf energy conservation}

\vspace{5mm}

But see Friday's lecture on ``conservation''...




\end{slide}

% -------------------------------------------------------------------------

\begin{slide}

\B{\bf The \( \rho \)-\( \theta \)-\( q \) conundrum}

\vspace{3mm}

\( \bullet \) Optimal wave propagation requires \( \rho \) staggered
with respect to \( \theta \)

\vspace{2mm}

\( \bullet \) Conservation of moisture requires \( \rho \) collocated
with \( q \)

\vspace{2mm}

\( \bullet \) Physical parameterizations require \( q \) collocated
with \( \theta \)



\end{slide}

% ----------------------------------------------------------------------------

\begin{slide}



\end{slide}

% ----------------------------------------------------------------------------

\begin{slide}


\end{slide}




% ----------------------------------------------------------------------------



% ----------------------------------------------------------------------------






\end{document}



%%%%%%%%%%%%%%%%%%%%%%%%%%%%%%%%%%%%%%%%%%%%%%%%%%%%%%%%%%%%%%%%%%%%%%%%%
