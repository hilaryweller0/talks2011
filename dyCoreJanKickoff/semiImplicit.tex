%%%%%%%%%%%%%%%%%%%%%%%%%%%%%%%%%%%%%%%%%%%%%%%%%%%%%%%%%%%%%%%%%
\begin{slide}{Semi-implicit versus Horizontally Explicit Time Stepping}
%%%%%%%%%%%%%%%%%%%%%%%%%%%%%%%%%%%%%%%%%%%%%%%%%%%%%%%%%%%%%%%%%

\begin{list0}

\item Highest winds speeds in the atmosphere $\sim \frac{1}{4}$ speed of sound\\
$\rightarrow$ use time stepping which is explicit in the vertical and implicit in the horizontal

\item This will degrade accuracy while potentially improving parallelisation

\item Compare parallel scaling and accuracy on up to 10K cores of HECToR

\item Down to 15km horizontal resolution

\item Limitations on results
    \begin{list1}
    \item implementation dependent
    \item architecture dependent
    \item only up to $\sim$10K cores
    \item only down to 15km resolution
    \item dependent on explicit time stepping scheme
    \end{list1}

\item So the results will not be conclusive -- just indicative

\item Plans to test on more cores at higher resolution and different architectures should depend on these results

\item Comparisons of accuracy and conservation should be more conclusive

\item Results cannot be extrapolated to 100K processors
    \begin{list1}
    \item weak scaling should continue
    \item other problems like fault tolerance become an issue
    \end{list1}

\end{list0}

\end{slide}
