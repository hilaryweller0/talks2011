%%%%%%%%%%%%%%%%%%%%%%%%%%%%%%%%%%%%%%%%%%%%%%%%%%%%%%%%%%%%%%%%%
\begin{slide}{New Test Cases to Excite Grid Scale Oscillations}
%%%%%%%%%%%%%%%%%%%%%%%%%%%%%%%%%%%%%%%%%%%%%%%%%%%%%%%%%%%%%%%%%

{\centering\large\bf
    Why?
}

\begin{list0}

\item New numerical schemes can appear good for simple test cases but perform badly once sub-grid scale parametrisations are added

\item Numerical analysis does not always find deficiencies with new schemes \\
-- you need to know what you are looking for

\item Sub-grid scale parametrisations can excite grid scale oscillations and computational modes

\item Try to mimic the effects of sub-grid scale parametrisations in exciting grid-scale oscillations before the effort of developing a 3D model with parametrisations

\end{list0}

{\centering\large\bf
    What?
}

\begin{list0}

\item Ask project partners what test cases they find
    \begin{list1}
    \item difficult
    \item discriminating
    \item tend to excite oscillations
    \end{list1}

\item Modify existing test cases to reduce the forcing size to one computational point. Eg:
    \begin{list1}
    \item Geostrophically balanced flow of SWEs with a one cell mountaian
    \item An under-resolved 2D vertical slice through a mountain
    \item Oscillating, single point forcing (Thuburn, Newton Institute scoping meeting talk, 2009)
    \item A one grid cell, 2D bouyant bubble
    \end{list1}

\end{list0}

{\bf Schemes to be used by the Met Office should undergo these tests}

\end{slide}



