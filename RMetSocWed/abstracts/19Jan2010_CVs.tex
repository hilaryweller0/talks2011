\documentclass[12pt]{article}

% margins >=2cm

\textwidth 17cm
\oddsidemargin -2mm
\textheight 25.6cm
\topmargin -27mm
\footskip 18pt

\parindent=0.0in
\parskip=12pt

\usepackage{graphicx, color, tabularx, setspace, url}
%\usepackage[T1]{fontenc}
%\usepackage[scaled]{uarial}
%\renewcommand*\familydefault{\sfdefault}
\renewcommand{\rmdefault}{phv} % Arial
\renewcommand{\sfdefault}{phv} % Arial
\usepackage{wrapfig}
%\usepackage[justification=centering]{caption,subfig}
\usepackage[justification=centering]{subfig}
\usepackage[round]{natbib}
\setlength{\bibsep}{0pt}
\setlength{\bibhang}{3pt}
%\renewcommand{\baselinestretch}{1.6}

% tabularx stuff
\newcolumntype{Y}{>{\raggedright\arraybackslash}X}
\renewcommand{\tabularxcolumn}[1]{>{\raggedright\arraybackslash}p{#1}}
%\renewcommand{\tabularxcolumn}[1]{>{\raggedright\arraybackslash}m{#1}}

%%% modifications to make article style more compact
\makeatletter

\def\@part[#1]#2
{
    \refstepcounter{part}
    {
        \parindent \z@ \raggedright \interlinepenalty \@M
        \normalfont \Large\bfseries
        \partname\nobreakspace\thepart : \nobreakspace #2 %\markboth{}{}\par
    }
    \nobreak \vskip 1.3ex \@afterheading
}
\renewcommand\section
{
    \@startsection {section}{1}{\z@}{1ex \@plus 2ex \@minus -.2ex}%
   {1ex \@plus 0.1ex}{\large\bfseries\raggedright}
}
\renewcommand\subsection
{
    \@startsection {subsection}{1}{\z@}{0.5ex \@plus 1ex \@minus -.2ex}%
   {0.1ex \@plus .1ex}{\bfseries\raggedright}
}
\renewcommand\subsubsection
{
    \@startsection {subsubsection}{1}{\z@}{-0.5ex \@plus -1ex \@minus -.2ex}%
   {0.1ex \@plus .1ex}{\normalfont\it\raggedright}
}
\renewcommand\paragraph{\@startsection{paragraph}{4}{\z@}%
                                    {1.25ex \@plus1ex \@minus.2ex}%
                                    {-0.4em}%
                                    {\normalfont\normalsize\it}}

\def\@maketitle
{
  \begin{center}%
  \let \footnote \thanks
    {\Large\bf \@title \par}%
    \vskip 0.5em%
    {\large
      \lineskip 1em%
      \begin{tabular}[t]{c}%
        \@author
      \end{tabular}\par}%
    \vskip 0.5em%
    {\large \@date}%
  \end{center}%
  \par
  %\vskip 1.5em
}

\renewenvironment{itemize}
   {\begin{list}{$\bullet$}{
      \setlength{\itemsep}{0pt}
      \setlength{\labelwidth}{1ex}
      \setlength{\leftmargin}{2ex}
      \setlength{\topsep}{0pt}
     }
   }
   {\end{list}}

\def\enumhook{}
\def\enumerate{%
  \ifnum \@enumdepth >\thr@@\@toodeep\else
    \advance\@enumdepth\@ne
    \edef\@enumctr{enum\romannumeral\the\@enumdepth}%
      \expandafter
      \list
        \csname label\@enumctr\endcsname
        {\usecounter\@enumctr\def\makelabel##1{\hss\llap{##1}}%
          \enumhook \csname enumhook\romannumeral\the\@enumdepth\endcsname}%
  \fi}

\renewcommand{\enumhook}
{
    \setlength{\topsep}{2pt}
    \setlength{\itemsep}{-4pt}
    \setlength{\labelwidth}{1ex}
    \setlength{\leftmargin}{3ex}
    \raggedright
}

\renewcommand{\descriptionlabel}[1]{\parbox{\leftmargin}{\raggedleft #1.~}}

\makeatother

\renewcommand{\floatpagefraction}{0.95}

\newcommand{\dprod} {\ensuremath{\,{\scriptscriptstyle \stackrel{\bullet}{{}}}\,}}
\renewcommand{\vec}[1] {\ensuremath{\mathbf #1}}
\newcommand{\eg}{{\it e.g.}}
\newcommand{\ie}{{\it i.e.}}
\newcommand{\de}{\ensuremath{^\circ}}

% command for generating the height and PV error plots for case #1
\newcommand{\hPVplots}[1]{
\begin{figure}[ht]
\begin{tabular}{cc}
Height (m) along 90\de\ longitude & Velocity (m/s) along 90\de\ longitude \\
\includegraphics[width=0.49\linewidth]{#1/hLon90.eps} &
\includegraphics[width=0.49\linewidth]{#1/UfLon90.eps} \\
Height (m) and velocity differences from initial &
Total vorticity and velocity differences from initial \\ & \\
\includegraphics[width=0.49\linewidth]{#1/hUfdiff.eps} &
\includegraphics[width=0.49\linewidth]{#1/PVdiff.eps} \\
\end{tabular}
\end{figure}
}

% command for generating the discretisation error plots for case #1
\newcommand{\discErrors}[1]{
\begin{figure}[ht]
\begin{tabular}{cc}
$\vec{U}$ and $\nabla \dprod \vec{U}$ & $h\vec{U}$ and $\nabla \dprod h\vec{U}$ \\
\includegraphics[width=0.49\linewidth]{#1/divu.eps} &
\includegraphics[width=0.49\linewidth]{#1/div.eps} \\
Discretisation error of $gh\nabla h$ normal to faces &
Discretisation error of $2h\vec{\Omega} \times \vec{U}$ normal to faces \\
\includegraphics[width=0.49\linewidth]{#1/ghGradhDiff.eps} &
\includegraphics[width=0.49\linewidth]{#1/hSfxFfdotUDiff.eps} \\
Discretisation error of $\nabla \dprod h\vec{U}\vec{U}$ normal to faces &
Resultant $\partial /\partial t (h\vec{U})$ ($m^2s^{-2}$) \\
\includegraphics[width=0.49\linewidth]{#1/divhUUDiff.eps} &
\includegraphics[width=0.49\linewidth]{#1/ddtPhi.eps}
\end{tabular}
\end{figure}

}
\begin{document}
\sloppy\raggedright


\begin{center}
\large\bf
    CVs \\
    Next generation modelling techniques for the \\atmosphere and ocean \\
    Wednesday 19 January 2010
\end{center}

{\bf Prof~Julia~Slingo}, Chief Scientist of the Met Office

{\bf Dr David Burridge CBE} \\
Dr. David Burridge has contributed to the scientific literature on numerical methods, physical parameterization, global modelling, data analysis and the diagnosis of the atmosphere. He was the Director of ECMWF from 1991 to 2004, consustant to the Scandinavian Weather Services and WMO since and is now Director of the WMO International Programme Office for THORPEX. He was President of the Royal Met Soc from 2000 to 2002 and President of the European Met Soc from 2005 to 2008.

{\bf Dr. Bill Skamarock} \\
Dr. Skamarock received his PhD from Stanford University in computational geophysical fluid dynamics from the departments of computer science and engineering. He is a Senior Scientist at NCAR where he has worked for the last 23 years after leaving Stanford. In addition to studies in atmospheric dynamics, convection and chemistry, Dr. Skamarock is very active in model development and is one of the principal architects of the Weather Research and Forecast (WRF) model.

{\bf Dr Matthew Piggott} \\
Matthew Piggott completed a PhD in numerical analysis from the University of Bath in 2001. He has been at Imperial College London since then and is now Grantham Reader in Ocean Modelling. He is based in the Dept. of Earth Science and Engineering and is attached to the Grantham Institute for Climate Change and the Applied Modelling and Computation Group. His research involves the development, analysis and application of numerical methods for geophysical and industrial fluids problems, and in particular methods based upon adaptive and unstructured meshes.

{\bf Prof John Thuburn} \\
During his post-doc at the University of Reading, John Thuburn was one of three ``Model Development Coordinators'' working on the UK Universities' Global Atmospheric Modelling Programme (UGAMP). After taking up a lectureship at Reading he began collaborating with the Met Office Dynamics Research group on numerical methods around 2000. In 2005 this collaboration strengthened when he took up a jointly funded chair at the University of Exeter. He has contributed particularly to the development of the Met Office ENDGame dynamical core. He also has collaborations with NCAR and Los Almos on numerical methods.

{\bf Dr Colin Cotter} \\
Dr Colin Cotter obtained a PhD on energy/momentum/potential vorticity/balance conserving numerical methods from the Department of Mathematics in Imperial College London. He then worked as a postdoctoral research associate, first with the Imperial College Ocean Model group in Earth Science and Engineering and then in Mathematics, both at Imperial College, working on modelling of internal waves in the South China Sea. Colin was appointed as a lecturer in the Aeronautics department at Imperial College in 2006, where he works on numerical methods for numerical weather prediction, ocean modelling and climate.

{\bf Dr Joanna Szmelter} \\
Joanna Szmelter gained her Ph.D. and worked as a RA at Swansea University. Following this, she was a Head of the Aerodynamic Technology Group at BAe Airbus Ltd., and then joined The Royal Military College of Science at Cranfield University, as a senior lecturer. In October 2006, she moved to Loughborough University where she is a senior lecturer at Wolfson School. Her research interests are in development of numerical methods for advanced scientific simulations.

{\bf Dr Hilary Weller} \\
Hilary Weller gained her PhD in 2002 and did her first post-doc on tropical climate variability in Meteorology at Reading University. Motivated by the shortfalls and inflexibility of state of the art climate models she started working independently on alternative techniques for numerical modelling of the atmosphere as an NCAS Climate core scientist at Reading. In 2010 she was awarded a NERC Advanced Fellowship to work on adaptive mesh modelling of the atmosphere at Reading. 

\end{document}
